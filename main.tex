\documentclass{article}
\usepackage{basic}
\newcommand{\anne}[2]{\colorbox{pink!75!blue}{#1}\marginpar[]{\tiny\textbf{\color{pink!50!blue}#2}}}

\title{Working title: Mirkovi\'c--Vybornov fusion in Beilinson--Drinfeld Grassmannian}
% \title{mvybdcluster}
% \author{Roger Bai, Anne Dranowski, Joel Kamnitzer}
\date{October 2020}

\begin{document}

\maketitle

\section{Introduction}

The BD Grassmannian. The convolution Grassmannian. Distinguished orbits, slices therein. Mirkovi\'c--Vybornov.

\section{Notation}

Let $\Gr$ denote the ordinary affine Grassmannian, $\Grbd$ the Beilinson--Drinfeld affine Grassmannian, and $\Grc$ the convolution affine Grassmannian. 

\begin{definition}
    Say $\mu_1$ and $\mu_2$ are \new{disjoint} if $(\mu_1)_i\ne 0 \Rightarrow (\mu_2)_i = 0$ and $(\mu_2)_i\ne 0 \Rightarrow (\mu_1)_i = 0$. 
\end{definition}

\acom{I propose ``anodyne'' as another candidate for the above property after Kapranov--Shechtman.}

\section{Main results}

\begin{claim}
$\widetilde{T_x^a}\to\pi^{-1}(\overline{\Gr^\lambda}\cap\Gr_\mu)$ (this does depend on $b$! we get something like a springer fibre where the action of [what] on either side has eigenvalues a permutation of $b$.)
\end{claim}

\begin{claim}
Let $\cW^\mu_{\rm BD} = G_1\xT[t^{-1}]t^\mu$. Then $S^{\mu_1 + \mu_2}$ is contained in $\cW^\mu_{\rm BD}$ if $\mu$ is dominant. \jcom{And $\mu_1$, $\mu_2$ are dominant also?} \acom{Roger has a proof.}
\end{claim}

\begin{claim}
Let $a = (0,s)$ and suppose $\mu_1$ and $\mu_2$ are disjoint \sout{``transverse''} 
% i.e.\ $(\mu_1)_i\ne 0 \Rightarrow (\mu_2)_i = 0$ and $(\mu_2)_i\ne 0 \Rightarrow (\mu_1)_i = 0$. 
Let $\mu = \mu_1 + \mu_2$. Then $X\in\widetilde{T_x^a}$ is a $\mu\times\mu$ block matrix, with $(\mu_1)_k\times(\mu_1)_k$ diagonal block conjugate to a $(\mu_1)_k$ Jordan block and $(\mu_2)_k\times (\mu_2)_k$ diagonal block conjugate to $(\mu_2)_k$ Jordan block plus $sI$.
\end{claim}

\begin{question}
If $\mu_i$ is not a permutation of $\lambda_i$ and $\lambda_i$ are not ``homogeneous'' how do we proceed? E.g.\ if $\mu_1 = (3,0,2)$, $\mu_2 = (0,2,0)$ and $\lambda_1 = (4,1)$, $\lambda_2 = (2,0,0)$. 
\end{question}

\begin{question}
If $\mu_1$ and $\mu_2$ are not disjoint how do we proceed? E.g.\ if $\mu_1 = (2,2,0)$, $\mu_2 = (1,0,2)$; $\mu_1 = (2,2,1)$, $\mu_2 = (1,0,1)$.
\end{question}

\section{Convolution vs BD}
% aka, our result vs MVy's results
% aka, what they do, and what they stop short of doing

Fix $G = \GL(U) \cong \GL_m\CC$ and $\{e_1,\dots,e_m\}$ a basis of $U$. 
% 
Recall $\Gr = G(\cK) / G(\cO)$ where $\cK,\cO$\dots

\begin{definition}[Beilinson--Drinfeld loop Grassmannians]
    Denoted $\Grbd_{C^{(n)}}$ with $C$ a smooth curve (or formal neighbourhood of a finite subset thereof) and $C^{(n)}$ its $n$th symmetric power. It is a reduced ind-scheme $\Grbd_{C^{(n)}}\to C^{(n)}$ with fibres of $C$-lattices $\Grbd_b = \{(b,\cL) : b \in C^{(n)}\}$ made up of vector bundles \anne{such that}{Not sure what $\cO_C$ means} $\cL \cong U\otimes\cO_C$ off $b$ (i.e.\ over $C - \underline b$). The standard \anne{lattice}{Notation} is the pair $(\varnothing,\cL_0)$ with $\cL_0 = U\otimes \cO_C$. 
\end{definition}

{\bf The case $n = 1$.} Fix $b\in C$ and $t$ a choice of formal parameter. \anne{Then}{Why is this called ``its group-theoretic realization''} $\Grbd_b\cong\Gr$.

Furthermore, in this case, $C$-lattices $(b,\cL)$ are identified with $\cO$-submodules $L = \Gamma(\hat b, \cL)$ of $U_\cK = U\otimes\cK$ such that $L\otimes_\cO\cK \cong U_\cK$. 

Under this identification, we associate to a given $\lambda\in\ZZ^m$ the lattice (a priori a $\cO$-submodule) $L_\lambda = \oplus_1^m t^{\lambda_i}e_i\cO$. Nb. our lattices will be contained in the standard lattice $L_0$ whereas MVy's lattices contain.

Connected components of $\Gr$ are 

% Distinguished orbits. 
$G(\cO)$-orbits are indexed by coweights $\lambda = (\lambda_1\ge\lambda_2\ge\cdots\ge\lambda_n)$ of $G$. In terms of lattices 
\begin{equation}
    \label{eq:grlambdalat}
    \Gr^\lambda = \left\{L\supset L_0 \,\big|\, t\big|_{L/L_0} \in \OO_\lambda \right\}
\end{equation} in the connected component $\Gr_N$ are indexed

\cite{mirkovic2007quiver} define a map 
\begin{equation}
    \label{eq:grbdingrc}
    \Grbd \to \Grc
\end{equation}

\begin{itemize}
    \item Their slice $T_x$ or $T_\lambda$
    \item Their embedding $T_x\to\mathfrak G_N$
    \item $N$-dim $D$
    \item The map $\tilde {\bf m}:\tilde{\mathfrak g}^{n}\to\End(D)$
    \item The map ${\bf m} : \tilde{\mathcal N}^n\to\mathcal N$ sending $(x,F_\bullet)$ to $x$
    \item The map $\pi: \tilde{\mathfrak G}^n\to \mathfrak G$ sending $\mathcal L_\bullet$ to $\mathcal L_n$
\end{itemize}


The special case $b = \vec 0$. In this case $0$ in the affine quiver variety goes to the point $L_\lambda$ in the affine Grassmannian, and the preimage of zero in the smooth quiver variety ($=$ the core?) is identified with the preimage of $L_\lambda$ in the BD Grassmannian. 
% 
\[
\begin{tikzcd}
    \mathfrak L(\vec v,\vec w) \ar[r] \ar[d] & \pi^{-1} (L_\lambda)\ar[d] \\
     0 \ar[r] & L_\lambda
\end{tikzcd}    
\]

MVy write: ``we believe that one should be able to generalize this to arbitrary $b$'' and that's where we come in!

Recall the Mirkovi\'c--Vybornov immersion \cite[Theorems 1.2 and 5.3]{mirkovic2007quiver}. 

% \acom{All this time it was only an immersion! So our result will be a Theorem and not simply a reformulation plus corollary. Never mind.}

\begin{theorem}(\cite[Theorem 1.2 and 5.3]{mirkovic2007quiver})
    There exists an algebraic immersion $\tilde\psi$ 
    $$\widetilde{\bf m}^{-1}(T_\lambda)\cap\tilde\g^{n,a,E,\tilde\mu}\xrightarrow{\tilde\psi}\tilde{\mathfrak G}^{n,a}_{b}(P)$$
\end{theorem}

% Where to begin? How to tell this story? Left to right? 

% Here
% \begin{itemize}
%     \item $T_\lambda \equiv \TT_\lambda$
%     \item 
% \end{itemize}

\section{Statements and Proofs of Results}
\acom{Maybe split for now into a Notation section and a Proofs section}

Define
\[
S_{\mu_1, \mu_2} = N((t^{-1}))t^{\mu_1}(t-s)^{\mu_2}
\]
and
\[
W_\mu = G_1 [[t^{-1}]]t^\mu.
\]
Let $|\lambda| = |\lambda_1 + \lambda_2|$ and $|\mu| = |\mu_1 + \mu_2|$.

\acom{Why not $\lambda = \lambda_1 + \lambda_2$ and recall $\lvert \nu \rvert$ in general.}

\begin{lemma}[Proof in Proposition 2.6 of KWWY]
Suppose $\mu$ is dominant. Then 
\[
N((t^{-1})) t^\mu = N_1[[t^{-1}]] t^\mu.
\]
\end{lemma}

\begin{lemma}
For dominant $\mu_1,\mu_2$, we have
\[
S_{\mu_1, \mu_2} \subset W_{\mu_1 + \mu_2}.
\]
\end{lemma}

\begin{proof}
We have
\[
\begin{split}
    S_{\mu_1, \mu_2} & = N((t^{-1}))t^{\mu_1}(t-s)^{\mu_2} \\
     & \subset T_1[[t^{-1}]] N((t^{-1})) t^{\mu_1} (t-s)^{\mu_2} \\
     & = T_1[[t^{-1}]] N_1[[t^{-1}]] t^{\mu_1} (t-s)^{\mu_2} \\
     & = B_1[[t^{-1}]] t^{\mu_1} (t-s)^{\mu_2} \\
     & = B_1[[t^{-1}]] t^{\mu_1 + \mu_2} \\
     & \subset G_1[[t^{-1}]] t^{\mu_1 + \mu_2} \\
     & = W_{\mu_1 + \mu_2}
\end{split}
\]
where $B_1[[t^{-1}]] t^{\mu_1} (t-s)^{\mu_2} = B_1[[t^{-1}]] t^{\mu_1 + \mu_2}$ since 
\[
\frac{t}{t-s} = 
1 + \frac{s}{t} + \frac{s^2}{t^2} + \cdots 
\in B_1[[t^{-1}]].
\]
\end{proof}

Define $\Gr^{\lambda_1, \lambda_2} \subset \Gr_{BD}$ to be the family with generic fibre $\Gr^{\lambda_1} \times \Gr^{\lambda_2}$ and 0-fibre $\Gr^{\lambda_1 + \lambda_2}$.

Define $\OO_{\lambda_1, \lambda_2}$ to be matrices $X$ of size $|\lambda| \times |\lambda|$ such that 
\[
    % \begin{cases}
        X\big|_{E_0} \in \OO_{\lambda_1} 
        % \\
        \text{ and }
        (X-sI)\big|_{E_s} \in \OO_{\lambda_2}
    % \end{cases}
\]

Let 
\[
\mu = (\mu^{(1)}, \mu^{(2)}, ..., \mu^{(n)}).
\]
Define $\TT_{\mu_1, \mu_2}$ to be $|\mu| \times |\mu|$ matrices $X$ such that $X$ consists of block matrices where the size of the $i$-th diagonal block is $|\mu^{(i)}| \times |\mu^{(i)}|$, for $1\leq i \leq n$. Each diagonal block is the companion matrix for $t^{\mu_1}(t-s)^{\mu_2}$.

\begin{theorem}
We have an isomorphism
\[
    \overline{\Gr^{\lambda_1, \lambda_2}} \cap S_{\mu_1, \mu_2} \cong
    \overline{\OO_{\lambda_1, \lambda_2}} \cap \TT_{\mu_1, \mu_2} \cap \n.
\]
\end{theorem}

\acom{Rather, corollary?}

\begin{proof}
We will prove this similarly to how the usual Mirkovi\'c--Vybornov isomorphism is proven.
\begin{enumerate}[label = Step \arabic*:]
    \item Define a map $\TT_{\mu_1, \mu_2} \cap \cN \rightarrow G_1[t^{-1}, (t-s)^{-1}] t^{\mu_1} (t-s)^{\mu_2}$.
    % 
    $$
    A \mapsto t^{\mu_1} (t-s)^{\mu_2} + a(t, t-s) \mapsto (L_1 \subset L_2) : (t-s)\big|_{L_2/L_1} = A\big|_{E_s}  , t\big|_{L_1/L_0} = A\big|_{E_0}
    $$
    % 
    Question: 1. is the middle matrix similar to a block matrix? 2. is the composition of these maps some intermediate level of MVy's $\psi$'s 

    BD Gr as lattices? $(L_1,L_2)\in\Gr\times\Gr$ corresponds to $L$ such that $L\otimes\CC\xt\cong L_1\otimes\CC\xt$ and $L\otimes\CC\xt[t-s]\cong L_2\otimes\CC\xt[t-s]$ where $\otimes = \otimes_{\CC[t]}$ or $\otimes_{\CC[t-s]}$ respectively even though Roger believes $\CC[t] = \CC[t-s]$.  

    \item If $A \in \TT_{\mu_1,\mu_2} \cap \n$ then $A$ is sent to $(N_-)_1[t^{-1}, (t-s)^{-1}] t^{\mu_1} (t-s)^{\mu_2}$. \acom{Requires MVyBD!}

    \item Conversely, given $L \in W_{\mu_1 + \mu_2}$, want to show surjectivity.
\end{enumerate}
\end{proof}

\include{examplesp}

\bibliographystyle{alpha}
\bibliography{mvybd}
\end{document}
