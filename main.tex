\documentclass{article}
\usepackage{basic}
\newcommand{\anne}[2]{\colorbox{pink!75!blue}{#1}\marginpar[]{\tiny\textbf{\color{pink!50!blue}#2}}}

\title{Working title: Mirkovi\'c--Vybornov fusion in Beilinson--Drinfeld Grassmannian}
% \title{mvybdcluster}
% \author{Roger Bai, Anne Dranowski, Joel Kamnitzer}
\date{October 2020}

\begin{document}

\maketitle

\section{Introduction}

The BD Grassmannian. The convolution Grassmannian. Distinguished orbits, slices therein. Mirkovi\'c--Vybornov \cite{mirkovic2007quiver,mirkovic2019comparison}, Cautis--Kamnitzer \cite{cautis2018categorical}, Anderson--Kogan \cite{anderson2005algebra}.

\section{Notation}

Let $\Gr$ denote the ordinary affine Grassmannian, $\Grbd$ the Beilinson--Drinfeld affine Grassmannian, and $\Grc$ the convolution affine Grassmannian. 

\begin{definition}
    The \new{BD Grassmannian} is the set 
    \begin{equation}
        \label{eq:grbd}
    \begin{aligned}
        \{(V,\sigma) : V &\text{ is a rank } m \text{ vector bundle on }\PP^1 \\ &\text{and } \sigma: V \dashrightarrow \scO_{\PP^1}^m \text{ is a trivialization} \\
        &\text{defined away from finitely many points in }\AA^1\}
    \end{aligned}
    \end{equation}
    More generally, one can define a BD grassmannian over any smooth curve $C$ as the reduced ind-scheme $\Grbd_C$ fibered over a finite symmetric power of $C$ such that the fibre over the point $\vec p$ is a collection of vector bundles over $C$ which are trivial away from $\vec p$ viewed also as a subset of $C$. To \cite{mirkovic2019comparison} the rank $m$ of the trivial fibres $\scO_C^m$ is the of the group $\GL_m\CC$. 
\end{definition}

To quote \cite{baumann2020bases} the BD Grassmannian is a relative version of the affine Grassmannian where the base is the space of effective divisors on a smooth curve. The choice of $\AA^1$ amply satisfies our needs and offers three advantages: there is a natural global coordinate it, every $G$-torsor on it is trivializable, and the monodromy of any local system is trivial. Formally, $\Grbd$ is the functor on the category of commutative $\CC$-algebras that assigns to an algebra $R$ the set of isomorphism classes of triples $(\vec p,V,\sigma)$ where $\vec p\in \AA^n(R)$, $V$ is a $G^\vee$-torsor over $\AA^1_R$ and $\sigma$ is a trivialization of $V$ away from $\vec p$. They denote by $\pi$ the fibration $\Grbd\to\AA^n$ (forgetting $V$ and $\sigma$). 

Their simplified description is: it's the set of pairs $(\vec p,[\sigma])$ where $\vec p\in\CC^n$ and $[\sigma]$ is an element of the homogeneous space 
\[
G^\vee(\CC[z,(z-p_1)^{-1},\dots,(z-p_n)^{-1}])/G^\vee(\CC[z])    
\]
Their example, our setting: 
\begin{example}
    When $G = \GL_m\CC$ the datum of $[\sigma]$ is equivalent to the datum of the $\CC[z]$-lattice $\sigma(L_0)$ in $\CC(z)^m$ with $L_0 = \CC[z]^m$ denoting the \new{standard lattice}. Set $f_{\vec p} = (z - p_1)\cdots (z-p_n)$. Then a lattice $L$ is of the form $\sigma(L_0)$ if and only if there exists a positive integer $k$ such that $f^k_{\vec p}(L_0) \subseteq L\subseteq f^{-k}_{\vec p}(L_0)$ and for each $k$ they denote by $\Grbd_k$ the subset of $\Grbd$ consisting of pairs $(\vec p,L)$ such that this sandwhich condition holds. They identify $\CC[z]/(f_{\vec p}^{2k})$ with the vector space of polynomials of degree strictly less than $2kn$, and $L_0/f_{\vec p}^{2k}L_0$ with its $N$th product. Then 
    \[
    \Grbd_k \overset{\text{Zariski closed}}{\subset} \CC^n\times\bigcup_{d=0}^{2knN} G_d(L_0/f_{\vec p}^{2k}L_0)  
    \]
    where $G_d(?)$ denotes the ordinary Grassmann manifold of $d$-planes in the argument.
\end{example}

\begin{definition}
    The \new{deformed convolution Grassmannian} is [not needed?]
    pairs $(\vec p,[\vec\sigma])$ where $\vec p\in\CC^n$ and $\vec\sigma$ is in 
    \[
    G^\vee(\CC[z,(z-p_1)^{-1}]) \times^{G^\vee(\CC[z])}\cdots\times^{G^\vee(\CC[z])} G^\vee(\CC[z,(z-p_n)^{-1}])/G^\vee(\CC[z])
    \] with a map down to $\Grbd$ defined by $(\vec p,[\vec \sigma])\mapsto (\vec p,[\sigma_1\cdots\sigma_n])$. 
\end{definition}
To steal the follow-up example in \cite{baumann2020bases} where the above definition is also copied from\dots 
\begin{example}
    When $G = \GL_m\CC$ this deformation is described by the datum of $\vec p\in\CC^n$ and a sequence $(L_1,\dots,L_n)$ of $\CC[z]$-lattices in $\CC(z)^m$ such that for some $k\in\ZZ$ and for all $j \in \{1\dots n\}$
    \[
    (z-p_j)^kL_{j-1}\subset L_j\subset (z-p_j)^{-k} L_{j-1}    
    \]
    where again $L_0 = \CC[z]^m$ denotes the standard lattice, and now $L_j = (\sigma_1\cdots\sigma_j)(L_0)$. Very nice. Very concrete. They can partition the deformation into \new{cells} by specifying the \new{relative positions} of the pairs $(L_{j-1},L_j)$ in terms of \new{invariant factors}. 
\end{example}
To be continued: \cite{baumann2020bases} go on to describe the fibres of the composition deformation to $\Grbd$ to $\CC^n$ and their description maybe helpful.

For $\mu\in P$ and $p\in\CC$ they define 
\[
    \tilde S_{\mu | p} = (z-p)^\mu N^\vee (\CC[z,(z-p)^{-1}]) = N^\vee (\CC[z,(z-p)^{-1}])(z-p)^\mu
\]
They note that $\CC\xT[z-p]$ is the completion of $\CC(z)$ at ``the place defined by p'' and identify $\CC\xt[z-p]$ with $\CC\xt[z]$ and $\CC\xT[z-p]$ with $\CC\xT[z]$. 

They claim that 
\[
    N^\vee (\CC[z,(z-p)^{-1}])/N^\vee (\CC[z]) \to N^\vee(\CC\xT[z-p])/N^\vee(\CC\xt[z-p]) \cong N^\vee(\cK)/N^\vee(\cO)
\]
is bijective, and that mapping $\Gr$ and multiplying by $(z-p)^\mu$ one gets 
\[
\tilde S_{\mu | p}/N^\vee(\CC[z])\cong S_\mu
\]
They go on to describe the fusion product (section 5.3) a probably worthwhile read. 

\begin{definition}
    Say $\mu_1$ and $\mu_2$ are \new{disjoint} if $(\mu_1)_i\ne 0 \Rightarrow (\mu_2)_i = 0$ and $(\mu_2)_i\ne 0 \Rightarrow (\mu_1)_i = 0$. 
\end{definition}

% \acom{I propose ``anodyne'' as another candidate for the above property after Kapranov--Shechtman.}

\section{Main results}

\begin{claim}
$\widetilde{T_x^a}\to\pi^{-1}(\overline{\Gr^\lambda}\cap\Gr_\mu)$ (this does depend on $b$! we get something like a springer fibre where the action of [what] on either side has eigenvalues a permutation of $b$.)
\end{claim}

\begin{claim}
Let $\cW^\mu_{\rm BD} = G_1\xT[t^{-1}]t^\mu$. Then $S^{\mu_1 + \mu_2}$ is contained in $\cW^\mu_{\rm BD}$ if $\mu$ is dominant. \jcom{And $\mu_1$, $\mu_2$ are dominant also?} \acom{Roger has a proof.}
\end{claim}

\begin{claim}
Let $a = (0,s)$ and suppose $\mu_1$ and $\mu_2$ are disjoint \sout{``transverse''} 
% i.e.\ $(\mu_1)_i\ne 0 \Rightarrow (\mu_2)_i = 0$ and $(\mu_2)_i\ne 0 \Rightarrow (\mu_1)_i = 0$. 
Let $\mu = \mu_1 + \mu_2$. Then $X\in\widetilde{T_x^a}$ is a $\mu\times\mu$ block matrix, with $(\mu_1)_k\times(\mu_1)_k$ diagonal block conjugate to a $(\mu_1)_k$ Jordan block and $(\mu_2)_k\times (\mu_2)_k$ diagonal block conjugate to $(\mu_2)_k$ Jordan block plus $sI$.
\end{claim}

\begin{question}
If $\mu_i$ is not a permutation of $\lambda_i$ and $\lambda_i$ are not ``homogeneous'' how do we proceed? E.g.\ if $\mu_1 = (3,0,2)$, $\mu_2 = (0,2,0)$ and $\lambda_1 = (4,1)$, $\lambda_2 = (2,0,0)$. 
\end{question}

\begin{question}
If $\mu_1$ and $\mu_2$ are not disjoint how do we proceed? E.g.\ if $\mu_1 = (2,2,0)$, $\mu_2 = (1,0,2)$; $\mu_1 = (2,2,1)$, $\mu_2 = (1,0,1)$.
\end{question}

\section{Convolution vs BD}
% aka, our result vs MVy's results
% aka, what they do, and what they stop short of doing

Fix $G = \GL(U) \cong \GL_m\CC$ and $\{e_1,\dots,e_m\}$ a basis of $U$. 
% 
Recall $\Gr = G(\cK) / G(\cO)$ where $\cK,\cO$\dots

\begin{definition}[Beilinson--Drinfeld loop Grassmannians]
    Denoted $\Grbd_{C^{(n)}}$ with $C$ a smooth curve (or formal neighbourhood of a finite subset thereof) and $C^{(n)}$ its $n$th symmetric power. It is a reduced ind-scheme $\Grbd_{C^{(n)}}\to C^{(n)}$ with fibres of $C$-lattices $\Grbd_b = \{(b,\cL) : b \in C^{(n)}\}$ made up of vector bundles \anne{such that}{Not sure what $\cO_C$ means} $\cL \cong U\otimes\cO_C$ off $b$ (i.e.\ over $C - \underline b$). The standard \anne{lattice}{Notation} is the pair $(\varnothing,\cL_0)$ with $\cL_0 = U\otimes \cO_C$. 
\end{definition}

{\bf The case $n = 1$.} Fix $b\in C$ and $t$ a choice of formal parameter. \anne{Then}{Why is this called ``its group-theoretic realization''} $\Grbd_b\cong\Gr$.

Furthermore, in this case, $C$-lattices $(b,\cL)$ are identified with $\cO$-submodules $L = \Gamma(\hat b, \cL)$ of $U_\cK = U\otimes\cK$ such that $L\otimes_\cO\cK \cong U_\cK$. 

Under this identification, we associate to a given $\lambda\in\ZZ^m$ the lattice (a priori a $\cO$-submodule) $L_\lambda = \oplus_1^m t^{\lambda_i}e_i\cO$. Nb. our lattices will be contained in the standard lattice $L_0$ whereas MVy's lattices contain.

Connected components of $\Gr$ are 

% Distinguished orbits. 
$G(\cO)$-orbits are indexed by coweights $\lambda = (\lambda_1\ge\lambda_2\ge\cdots\ge\lambda_n)$ of $G$. In terms of lattices 
\begin{equation}
    \label{eq:grlambdalat}
    \Gr^\lambda = \left\{L\supset L_0 \,\big|\, t\big|_{L/L_0} \in \OO_\lambda \right\}
\end{equation} in the connected component $\Gr_N$ are indexed

\cite{mirkovic2007quiver} define a map 
\begin{equation}
    \label{eq:grbdingrc}
    \Grbd \to \Grc
\end{equation}

\begin{itemize}
    \item Their slice $T_x$ or $T_\lambda$
    \item Their embedding $T_x\to\mathfrak G_N$
    \item $N$-dim $D$
    \item The map $\tilde {\bf m}:\tilde{\mathfrak g}^{n}\to\End(D)$
    \item The map ${\bf m} : \tilde{\mathcal N}^n\to\mathcal N$ sending $(x,F_\bullet)$ to $x$
    \item The map $\pi: \tilde{\mathfrak G}^n\to \mathfrak G$ sending $\mathcal L_\bullet$ to $\mathcal L_n$
\end{itemize}


The special case $b = \vec 0$. In this case $0$ in the affine quiver variety goes to the point $L_\lambda$ in the affine Grassmannian, and the preimage of zero in the smooth quiver variety ($=$ the core?) is identified with the preimage of $L_\lambda$ in the BD Grassmannian. 
% 
\[
\begin{tikzcd}
    \mathfrak L(\vec v,\vec w) \ar[r] \ar[d] & \pi^{-1} (L_\lambda)\ar[d] \\
     0 \ar[r] & L_\lambda
\end{tikzcd}    
\]

MVy write: ``we believe that one should be able to generalize this to arbitrary $b$'' and that's where we come in!

Recall the Mirkovi\'c--Vybornov immersion \cite[Theorems 1.2 and 5.3]{mirkovic2007quiver}. 

% \acom{All this time it was only an immersion! So our result will be a Theorem and not simply a reformulation plus corollary. Never mind.}

\begin{theorem}(\cite[Theorem 1.2 and 5.3]{mirkovic2007quiver})
    There exists an algebraic immersion $\tilde\psi$ 
    $$\widetilde{\bf m}^{-1}(T_\lambda)\cap\tilde\g^{n,a,E,\tilde\mu}\xrightarrow{\tilde\psi}\tilde{\mathfrak G}^{n,a}_{b}(P)$$
\end{theorem}

% Where to begin? How to tell this story? Left to right? 

% Here
% \begin{itemize}
%     \item $T_\lambda \equiv \TT_\lambda$
%     \item 
% \end{itemize}

\section{Statements and Proofs of Results}
\acom{Maybe split for now into a Notation section and a Proofs section}

Define
\[
S_{\mu_1, \mu_2} = N((t^{-1}))t^{\mu_1}(t-s)^{\mu_2}
\]
and
\[
W_\mu = G_1 [[t^{-1}]]t^\mu.
\]
Let $|\lambda| = |\lambda_1 + \lambda_2|$ and $|\mu| = |\mu_1 + \mu_2|$.

\acom{Why not $\lambda = \lambda_1 + \lambda_2$ and recall $\lvert \nu \rvert$ in general.}

\begin{lemma}[Proof in Proposition 2.6 of KWWY]
Suppose $\mu$ is dominant. Then 
\[
N((t^{-1})) t^\mu = N_1[[t^{-1}]] t^\mu.
\]
\end{lemma}

\begin{lemma}
For dominant $\mu_1,\mu_2$, we have
\[
S_{\mu_1, \mu_2} \subset W_{\mu_1 + \mu_2}.
\]
\end{lemma}

\begin{proof}
We have
\[
\begin{split}
    S_{\mu_1, \mu_2} & = N((t^{-1}))t^{\mu_1}(t-s)^{\mu_2} \\
     & \subset T_1[[t^{-1}]] N((t^{-1})) t^{\mu_1} (t-s)^{\mu_2} \\
     & = T_1[[t^{-1}]] N_1[[t^{-1}]] t^{\mu_1} (t-s)^{\mu_2} \\
     & = B_1[[t^{-1}]] t^{\mu_1} (t-s)^{\mu_2} \\
     & = B_1[[t^{-1}]] t^{\mu_1 + \mu_2} \\
     & \subset G_1[[t^{-1}]] t^{\mu_1 + \mu_2} \\
     & = W_{\mu_1 + \mu_2}
\end{split}
\]
where $B_1[[t^{-1}]] t^{\mu_1} (t-s)^{\mu_2} = B_1[[t^{-1}]] t^{\mu_1 + \mu_2}$ since 
\[
\frac{t}{t-s} = 
1 + \frac{s}{t} + \frac{s^2}{t^2} + \cdots 
\in B_1[[t^{-1}]].
\]
\end{proof}

Define $\Gr^{\lambda_1, \lambda_2} \subset \Gr_{BD}$ to be the family with generic fibre $\Gr^{\lambda_1} \times \Gr^{\lambda_2}$ and 0-fibre $\Gr^{\lambda_1 + \lambda_2}$.

Define $\OO_{\lambda_1, \lambda_2}$ to be matrices $X$ of size $|\lambda| \times |\lambda|$ such that 
\[
    % \begin{cases}
        X\big|_{E_0} \in \OO_{\lambda_1} 
        % \\
        \text{ and }
        (X-sI)\big|_{E_s} \in \OO_{\lambda_2}
    % \end{cases}
\]

Let 
\[
\mu = (\mu^{(1)}, \mu^{(2)}, ..., \mu^{(n)}).
\]
Define $\TT_{\mu_1, \mu_2}$ to be $|\mu| \times |\mu|$ matrices $X$ such that $X$ consists of block matrices where the size of the $i$-th diagonal block is $|\mu^{(i)}| \times |\mu^{(i)}|$, for $1\leq i \leq n$. Each diagonal block is the companion matrix for $t^{\mu_1}(t-s)^{\mu_2}$. Each off-diagonal block is zero everywhere except possibly in the last $\min(\mu_i,\mu_j)$ columns of the last row. 

\begin{theorem}
We have an isomorphism
\[
    \overline{\Gr^{\lambda_1, \lambda_2}} \cap S_{\mu_1, \mu_2} \cong
    \overline{\OO_{\lambda_1, \lambda_2}} \cap \TT_{\mu_1, \mu_2} \cap \n.
\]
\end{theorem}

\acom{Rather, corollary?}

\begin{proof}
We will prove this similarly to how the usual Mirkovi\'c--Vybornov isomorphism is proven.
\begin{enumerate}[label = Step \arabic*:]
    \item Define a map $\TT_{\mu_1, \mu_2} \cap \cN \rightarrow G_1[t^{-1}, (t-s)^{-1}] t^{\mu_1} (t-s)^{\mu_2}$.
    % 
    $$
    A \mapsto t^{\mu_1} (t-s)^{\mu_2} + a(t, t-s) \mapsto (L_1 \subset L_2) : (t-s)\big|_{L_2/L_1} = A\big|_{E_s}  , t\big|_{L_1/L_0} = A\big|_{E_0}
    $$
    % 
    Question: 1. is the middle matrix similar to a block matrix? 2. is the composition of these maps some intermediate level of MVy's $\psi$'s 

    BD Gr as lattices? $(L_1,L_2)\in\Gr\times\Gr$ corresponds to $L$ such that $L\otimes\CC\xt\cong L_1\otimes\CC\xt$ and $L\otimes\CC\xt[t-s]\cong L_2\otimes\CC\xt[t-s]$ where $\otimes = \otimes_{\CC[t]}$ or $\otimes_{\CC[t-s]}$ respectively even though Roger believes $\CC[t] = \CC[t-s]$.  

    \item If $A \in \TT_{\mu_1,\mu_2} \cap \n$ then $A$ is sent to $(N_-)_1[t^{-1}, (t-s)^{-1}] t^{\mu_1} (t-s)^{\mu_2}$. \acom{Requires MVyBD!}

    \item Conversely, given $L \in W_{\mu_1 + \mu_2}$, want to show surjectivity.
\end{enumerate}
\end{proof}

% 2020-12-13 19:13:45
Last meeting's todos:
\begin{itemize}
    \item make sure that the image of our map is in the $G_1$ orbit
    \item more generally, define the map, check that the map is well-defined
    \item Anne: say what little a is, i.e.\ insert the MVy theorem as stated in CK, or thesis
    \item Roger: check it
\end{itemize}

% !TeX root = ./main.tex

\section{Examples}

\begin{example}
$\lambda_1 = (1,0,0)$, $\lambda_2 = (1,1,0)$, $\mu_1 = (0,1,0)$, $\mu_2 = (1,0,1)$. \jcom{$\mu = \mu_1 + \mu_2$ determines the blocks we have on the RHS of the BD MVy isomorphism of Equation~\ref{eq:topmvy2}.} 

In the non-BD case, MVy establish
$$
\overline{\Gr^\lambda}\cap \cW^\mu \to 
\left\{ X = 
\left[\begin{BMAT}{cc|cc|c}{cc|cc|c}
0 & 1 & 0 & 0 & 0 \\
* & * & * & * & * \\
0 & 0 & 0 & 1 & 0 \\
* & * & * & * & * \\
* & 0 & * & 0 & * 
\end{BMAT}\right] \big| X \in \overline{\OO}_\lambda
\right\}
$$

In the BD case the RHS will consist of the same block like matrices $X$ but now having eigenvalues $s, 0$ such that $X - s\big|_{E_s}\in \OO_{\lambda_2}$ 
\end{example}

\begin{example}
Do Joel's exercise: It would be good to do an example where we will see some multiplicity in the fusion product.  I think that the simplest example would be the fusion product of the MV cycles for $\SL_3$ of weights $2\alpha_1$ and $2\alpha_2$.  Following the notation from the mvbasis paper, this would correspond to the following multiplication:
$x^2 y^2 = (xy-z)^2 + 2(xy - z) z + z^2$. 
For this example, I think we need $\lambda_1 = (2,0,0)$, $\lambda_2 = (2,2,0)$, $\mu_1 = (0,2,0)$, $\mu_2 = (2,0,2)$. 
\end{example}

\begin{example}
    Let 
    \[
    \begin{aligned}
        \mu_1 &= (2) & \lambda_1 &= & \mu &= (5) \\
        \mu_2 &= (3) & \lambda_2 &= & \lambda &= 
    \end{aligned}
    \]
    Consider the companion matrix $C(p)$ of 
    $$p(t) = (t-s)^3t^2 = (t^3 - 3t^2 s + 3t s^2 - s^3)t^2 = t^5 - 3 t^4 s + 3t^3 s^2 - t^2 s^3$$ 
    Let $X = C(p)^T$ so 
    \[
    X = \left[\begin{BMAT}(e){ccccc}{ccccc}
        0 & 1 & & & \\ 
        & 0 & 1 & & \\
        &   & 0 & 1 & \\ 
        &   &   & 0 & 1 \\
       0 & 0 & s^3 & -3s^2  & 3s  
    \end{BMAT}\right]    
    \]
    Ask that $X\big|_{E_0}$ has Jordan type $\lambda_1$ and $X-s\big|_{E_s}$ has Jordan type $\lambda_2$. In this rank 1 case we are forced to take $\lambda_i = \mu_i$. 
    
    So what are the generalized eigenspaces $E_i$ ($i = 1,2$)? Note $\dim E_0 = 2$ and $\dim E_s = 3$. 
    
    \acom{The basis
    \[
     [1], [t], [\bf 1], [\bf t], [\bf t^2]    
    \] 
    with $t[t] = 0$ and $t[{\bf t^2}] =   s^3 [{\bf 1}] - 3s^2 [{\bf t}] + 3s[{\bf t^2}]$ \textit{is not the correct basis to consider}.
    Hence my confusion of yore: what we would like is $t[t] = 0$ no? what the matrix is telling us is that $t[t] = [{\bf 1}]$. Can we still speak of \textit{two} generalized eigenspaces?}  % [3st^2 - 3s^2 t + s^3]

    Rather, take $B$ to be the basis $b_1 = e_1$, $b_2 = e_2$, $b_5 = e_5$, $b_4 = Xe_5$, $b_3 = X^2e_5)$. In this basis
    \[
    X_B = [X(b_i)] = \begin{bmatrix}
        0 & 1 & 0 & 0 & 0 \\
        0 & 0 & 0 & 0 & 0 \\
        0 & 0 & s & 1 & 0 \\
        0 & 0 & 0 & s & 1 \\
        0 & 0 & 0 & 0 & s
    \end{bmatrix}
    \]

    % I think this demonstrates why Roger's idea is probably the right thing to do. Namely, consider instead $T_{\mu_1,\mu_2}$ whose elements are 
    % \[
    %     X = \left[\begin{BMAT}(e){cc|ccc}{cc|ccc}
    %         0 & 1 & & & \\ 
    %         & 0 & x & y & z \\
    %         &   & 0 & 1 & \\ 
    %         &   &   & 0 & 1 \\
    %        0 & 0 & s^3 & -3s^2  & 3s  
    %     \end{BMAT}\right]    
    % \] 
    % such that $T_{\mu_1,\mu_2}\big|_{s = 0} = T_\mu$ whatever that means... 
\end{example}

\begin{example}
    Let 
    \[
    \begin{aligned}
        \mu_1 &= (3,1,1) & \lambda_1 &= (3,2,0) & \mu = (3,3,1) \\
        \mu_2 &= (0,2,0) & \lambda_2 &= (2,0,0) & \lambda = (5,2,0)
    \end{aligned}    
    \]
    and consider the companion matrices of 
    $$
    \begin{aligned}
        p_1(t) &= t^3 & p_2(t) &= t(t-s)^2 = t^3 - 2st^2 + s^2 t & p_3(t) &= t
    \end{aligned} % = t(t^2 - 2st + s^2) 
    $$
    \[
    X = \left[
        \begin{BMAT}(e){ccc;ccc;c}{ccc;ccc;c}
            0 & 1 & 0 & 0 & 0 & 0 & 0 \\
            0 & 0 & 1 & 0 & 0 & 0 & 0 \\
            % a & b & c & d & e & f & g \\ 
            0 & 0 & 0 & d & e & f & g \\ 
            0 & 0 & 0 & 0 & 1 & 0 & 0 \\
            0 & 0 & 0 & 0 & 0 & 1 & 0 \\
            0 & 0 & 0 & 0 & -s^2 & 2s & k \\
            0 & 0 & 0 & 0 & 0 & 0 & 0
        \end{BMAT}
        \right]    
    \]
\end{example}

\begin{example}
    Let
    \[
    \begin{aligned}
        \lambda_1 &= (3,2,0) & \mu_1 &= (3,1,1) \\
        \lambda_2 &= (2,0,0) & \mu_2 &= (1,1,0)
    \end{aligned}
    \]
    so the first MV cycle $Z_1 \cong \PP^1$ has MV polytope $\Conv\{0, \alpha_1\}$ and the second MV cycle $Z_2 \cong \PP^1$ has MV polytope $\Conv\{0, \alpha_2\}$. Their fusion product corresponds to two $\PP^2$'s intersecting along a $\PP^1$. That is, we have the fusion product
    \[
    Z_1 * Z_2 = Z_+ + Z_-
    \]
    where $Z_+ \cong Z_- \cong \PP^2$.
    We have 
    \[
    X = \begin{bmatrix}
    0 & 1 \\
      & 0 & 1 \\
      &   & 0 & 1 \\
      &   &   & s & a & b & c \\
      &   &   &   & 0 & 1 & 0 \\
      &   &   &   & 0 & s & d \\
      &   &   &   &   &   & 0
    \end{bmatrix}
    \]
    The $0$-generalized eigenspace $E_0$ of $X$ is $5$-dimensional, containing a 3-cycle and a 2-cycle.
    The 3-cycle is 
    \[
    \{X^2 e_3 , X e_3, e_3\} = \{e_1, e_2, e_3\}.
    \]
    To obtain another vector in $\ker X$, either $a = 0$ or $c = d = 0$, but the latter case cannot give a 2-cycle as $e_7 \notin \im X$. Then $a = 0$ and we obtain a 2-cycle
    \[
    \left\{X \left( e_6 -\frac{s}{d}e_7 \right), e_6 - \frac{s}{d}e_7 \right\}
    = \left\{e_5, e_6 - \frac{s}{d}e_7 \right\}.
    \]
    We also obtain the equations $b \neq 0$, $d \neq 0$, and $sc - bd = 0$ from this.
    
    For the $s$-generalized eigenspace $E_s$, we need $a + sb \neq 0$ to obtain a 2-cycle, which can be taken as 
    \[
    \begin{split}
    & \left\{ 
    (X -sI) \left( e_2 + 2s e_3 + 3s^2 e_4 + \frac{s^3}{a+sb} e_5 + \frac{s^4}{a+sb} e_6 \right), \right. \\ 
    & \quad \qquad \qquad  \left. e_2 + 2s e_3 + 3s^2 e_4 + \frac{s^3}{a+sb} e_5 + \frac{s^4}{a+sb} e_6 
    \right\} \\
    = & \left\{ 
    e_1 + s e_2 + s^2 e_3 + s^3 e_4, 
    e_2 + 2s e_3 + 3s^2 e_4 + \frac{s^3}{a+sb} e_5 + \frac{s^4}{a+sb} e_6
    \right\}
    \end{split}
    \]
    The minimal polymonial is $X^3 (X-sI)^2$, which when equated to 0 gives again the equation $cs-bd = 0$.
    Thus the defining equations are 
    \[
    \{ a = 0, cs-bd = 0\}.
    \]
    When we take $s = 0$, we get the equations
    \[
    \{a = 0, bd = 0\}
    \]
    which corresponds to two $\AA^2$'s intersecting along an $\AA^1$. This is indeed an open subset of $\PP^2 \cup_{\PP_1} \PP^2$, as required.
\end{example}

\begin{example}[Example 7 continued\dots]
    The matrix $X$ from the previous example defines (under MVy) the matrix 
    \[
        g = \begin{bmatrix}
            (t-s)t^3 \\
            -bt & (t-s)t \\
            -c & -d & t 
        \end{bmatrix}
    \]
    in $G(\cO)$. 
    Indeed the various blocks of $X$ are in a precise sense the companion matrices of the polynomial entries of $g$
    % First let's double check the inverse map. 

    In $\Gr$ the element $g$ defines the lattice 
    \[
        gL_0 = \CC[t]\langle
        (t-s)t^3e_1 - (a+bt) e_2 - ce_3 , (t-s)te_2 - de_3, te_3 
        \rangle
    \]
    and computing the matrix of the action of $t$ on the quotient $L_0/L$ in the basis 
    \[
    \{
        [e_1],[te_1],[t^2e_1],[t^3e_1],[e_2],[te_2],[e_3]
    \}    
    \]
    recovers $X$ up to a transpose of course. 

    Now let's see what we get when we invert $t$ and $t-s$ respectively. 

    First let's invert $t$ by considering $L_2 = L\otimes\CC\xt[t-s]$. 
    \[
        L_2  = \CC[t,t^{-1}]\langle(t-s) e_1 -\frac{a+bt}{t^3} e_2 - \frac c {t^3} e_3, (t-s) e_2 - \frac d t e_3, e_3 \rangle
    \]
    so in $L_0/L_2$ we have 
    \[
    t[e_1] = s[e_1] + \frac{a+bt}{t^3} [e_2] + \frac c {t^3}  [e_3] \qquad t[e_2] = s[e_2] + \frac d t[e_3]    \qquad [e_3] = 0 
    \]
    and
    \[
    \left[t\big|_{L_0/L_2}\right]_{\{[e_1],[e_2]\}} = \begin{bmatrix}
        s \\
        \frac{a+bt}{t^3} & s 
    \end{bmatrix}
    \]
    which upon subtracting $s I$ gives a matrix having block type $\mu_2$ and Jordan type $\lambda_2 = (2)$ assuming $\frac{a+bt}{t^3}\ne 0$. 
    
    Next let's invert $t-s$ by considering $L_1 = L\otimes\CC\xt$. 
    \[
        L_1 = \CC[t,(t-s)^{-1}]\langle
        t^3 e_1 - \frac{a+bt}{t-s}e_2 - \frac c {t-s} e_3, te_2 - \frac d {t-s} e_3 , te_3 
        \rangle 
    \]
    so in $L_0/L_1$ we have 
    \begin{align*}
        t[e_1] &= [te_1] \\
        t[te_1] &= [t^2 e_1] \\ 
        t[t^2e_1] & = \frac{a+bt}{t-s}[e_2] + \frac c {t-s} [e_3] \\
        t[e_2] &= \frac d {t-s} [e_3] \\ 
        t[e_3] &= 0 
    \end{align*}
    and 
    \[
    \left[t\big|_{L_0/L_1}\right]_{\{[e_1],[te_1],[t^2e_1],[e_2],[e_3]\}} = \begin{bmatrix}
        0 \\
        1 & 0 \\
        0 & 1 & 0 \\
          & & \frac{a+bt}{t-s} & 0 \\
          & & \frac c {t-s} & \frac d {t-s} & 0  
    \end{bmatrix}
    \]
    taking transpose 
    \[
        \begin{bmatrix}
            0 & 1\\
              & 0 & 1\\
              &   & 0 & \frac{a+bt}{t-s} & \frac c {t-s} \\
              & & & 0 & \frac d {t-s} \\
              & & & & 0  
        \end{bmatrix}
    \]
    which has block type $\mu_1$ and Jordan type $\lambda_1 = (3,2)$ assuming $\frac{a+bt}{t-s}=0$ and one of the entries in the last column is not zero!?

    We found $a = 0$ and $cs - bd = 0$ above. So actually $\frac{a+bt}{t^3} = \frac b{t^2} \ne 0$ and $\frac{a + bt}{t-s} = \frac{bt}{t-s}$...Roger find my mistake. 
\end{example}

\bibliographystyle{alpha}
\bibliography{mvybd}
\end{document}
