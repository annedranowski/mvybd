\documentclass{article}
\usepackage{basic}
% Working title: 
\title{How to compute the fusion product of MV cycles in type A}
\author{Roger Bai, Anne Dranowski, Joel Kamnitzer} 
\date{October 2020}

\begin{document}

\maketitle

\section{Introduction}
% 
The Mirkovic--Vybornov isomorphism \cite{mirkovic2007quiver} relates slices in the affine Grassmannian to slices in conjugacy classes of nilpotent matrices and Nakajima quiver varieties. As an application it provides a geometric version of symmetric and skew Howe $(\GL_m,\GL_n)$ dualities. 
% 

The Beilinson--Drinfeld Grassmannian, essential to \cite{mirkovic2007geometric}, is a space which is used to define a fusion (aka convolution) product on sheaves (representations) and MV cycles (bases). 
% 
This product structure makes the vector space spanned by the MV cycles into a commutative algebra isomorphic to the ring of functions on the unipotent subgroup (conjectured in \cite{anderson2003polytope} and proved in \cite{baumann2020bases}). % conjectured by JEA and proved by BKK
% 

The purpose of this paper is to give an elementary way to compute this product in type $A$. We do not know how fruitful this work might be in the large landscape of geometric representation theory as it is plagued by two major limitations: it does not readily generalize outside of type $A$, and it is as yet computationally expensive. 

\section{Players}

Fix $G = \GL_m$ and let $T\subset G$ be the maximal torus of diagonal matrices. 
% are there other possibilities?? :S 
Write $P$ for the character lattice of $T$. Identify $P\cong \ZZ^m$ so that the set of dominant weights $P_+$ is described/given by the set of partitions having at most $m$ nonzero parts. Set $\cO = \CC\xt$ and $\cK = \CC\xT$. We will also need the notation $\cO_x$ for the completion of the local ring $\CC[t]$ at $x\in\spec\CC[t]$ and $\cK_x$ for its fraction field $\CC\xT[t-x]$. 
% 
Note that (by choosing the local coordinate $t-x$ on $\spec\CC[t]$) we can identify $\cO_x \cong \cO$ and $\cK_x\cong\cK$. 

% Set $\AA = \AA^1_x$ (the subscript denoting the chosen local coordinate) and fix $s\in \AA - \{0\}$. 

\begin{itemize}
    \item The ordinary affine Grassmannian $\Gr = G(\cK)/G(\cO)$ % switched out ordinary Gr notation for curly Gr 
    \item The Beauville--Laszlo Grassmannian $\Grbd[x] = G(\cK_x)/G(\cO_x)$ % TODO: in terms of bundles it is $\{(P,\sigma) : P is a PGB over A and \sigma is a trivialization of P on A - x\}$
    \item The (two point) Beilinson--Drinfeld Grassmannian $\pi : \Grbd\to \AA$ (with one point fixed at 0) % switched out general curve $C$ for affine line $\AA$; TODO: in terms of bundles it is $\{(P,\sigma,x):agaiin P is a PGB on A, x \in A, and now \sigma is a trivialization away from 0 _and_ x \}
    \item Partitions $\mu_i\le\lambda_i$ of $N_i$ ($i=1,2$) and $\mu = \sum \mu_i \le \lambda = \sum\lambda_i$ of $N = \sum N_i$
    \item The slices $\Gr_\mu$ and $\cW_{\mu_1,\mu_2}$ to the orbits $\Gr^\lambda$ and $(\Grbd[2])^{\lambda_1,\lambda_2}$
    \item The nilpotent and semi-nilpotent cones $\cN$ and $\cN_s$ (of matrices with eigenvalues 0 and 0 or $s\ne 0$)
    \item The slices $\TT_\mu$ and $\TT_{\mu_1,\mu_2}$ to the orbits $\OO_\lambda$ and $\OO_{\lambda_1,\lambda_2}$ 
\end{itemize}

New (?) definitions among these are as follows.
% 

%The [family of] slices [with $s$-fibre?]
%\begin{equation}
%    \cW_{\mu_1,\mu_2} = G_1 [t^{-1},(t-s)^{-1}]L_{\mu_1,\mu_2}
%\end{equation}


We define
$$L_\mu = t^\mu, L_{\mu_1, \mu_2} = t^{\mu_1} (t-s)^{\mu_2} \in G((t^{-1}))/G[t]$$

\begin{equation}
    \cW_\mu = G_1[[t^{-1}]]L_\mu \subset G((t^{-1}))/G[t]
\end{equation}
a subscheme of the thick affine Grassmannian

Note that $ L_{\mu_1, \mu_2} = (t-s)^{\mu_2} t^{-\mu_2} L_{\mu} \in \cW_\mu$
and that  $L_{\mu_1,\mu_2}\in\Grbd[2]$ is a $\CC[t]$-lattice in $\CC(t)^m$ that specializes to a $\CC\xt$-lattice in $\CC\xT^m$ away from $t = 0$ and away from $t = s$; i.e.\
\begin{equation}
    \label{eq:defWBD}
    \begin{split}
        %L_{\mu_1,\mu_2}\otimes \CC[(t-s)^{-1}] =
        L_{\mu_1,\mu_2}\otimes \CC\xt &= L_{\mu_2} \otimes \CC\xt \text{ and } \\
        %L_{\mu_1,\mu_2}\otimes\CC[t^{-1}] =
        L_{\mu_1,\mu_2}\otimes \CC\xt[t-s] &= L_{\mu_1} \otimes \CC\xt[t-s]
    \end{split}
\end{equation}
where $L_{\mu_i}$ denotes the point $t^{\mu_i}G(\cO)\in\Gr$. 
% 

The family of semi-infinite orbits 
\begin{equation}
    S_{\mu_1,\mu_2} = N_-(\CC((t^{-1}))) L_{\mu_1,\mu_2}\,. 
\end{equation}

The orbit of $\CC[t]$-lattices in $\CC(t)^m$ whose elements specialize again to $\CC\xt$-lattices in $\CC\xT^m$ away from $t = 0$ and away from $t = s$ as follows
\begin{equation}
    \label{eq:defGrBDlambda}
    \begin{split}
        (\Grbd[2])^{\lambda_1,\lambda_2} = \{L \in\Grbd[2] : 
        %L\otimes\CC[t^{-1}] 
        L \otimes_{\CC[t]} \CC\xt[t-s] 
        &\in \Gr^{\lambda_2} \text{ and } \\
        % L\otimes\CC[(s-t)^{-1}] 
        L\otimes_{\CC[t]} \CC\xt &\in \Gr^{\lambda_1} \text{ and } \\ 
        L \otimes_{\CC[t]} \CC\xt[t-a] = L_0 \text{ for all $ a \ne 0, s$} \}\,.
    \end{split}
\end{equation}

\jcom{It is important to add the condition that the lattices are trivial at $ a $.}
% 
\acom{These defining conditions are again just telling us that in the fibre over the fixed point $(0,s)\in C^{(2)}$ this set is pairs of lattices; note that we invert the indeterminates $t$ and $(t-s)$ \textit{after} specializing $(0,s)$. Could we give a more explicit characterization like $t^{N_1}(t-s)^{N_2} L_0 \subseteq L\subseteq t^{-N_1}(t-s)^{-N_2}$?}

The [family ``$\cN\otimes\CC[s]$? $\cN\times\CC$?'' of] semi-nilpotent cone[s fibred over $C = \AA^1$ with $s$-fibre] 
\begin{equation}
    \cN_s = \{A \in \mat(N) : \text{eigenvalues of } A \text{ are } 0 \text{ or }s\}\,.
\end{equation}

The [family of] slice[s $\TT_{\mu_1,\mu_2}$ fibred over $C=\AA^1$ with $s$-fibre]
\begin{equation}
    \begin{split}
        \TT_{\mu_1,\mu_2}^s = \{ B + C_s: B &\text{ is a }\mu\times\mu \text{ block matrix of zeros} \\
        &\text{except possibly in the last }\min(\mu_i,\mu_j) \\ 
        &\text{columns of the last row of each }\mu_i\times\mu_j \text{ block} \\
        &\text{and } C_s \text{ is the block diagonal matrix of}\\
        &\text{companion matrices of } t^{\mu_{1,k}}(t-s)^{\mu_{2,k}}\}\,.
    \end{split}
\end{equation}

The uppertriangular subfamily $\TT_{\mu_1,\mu_2}^{+}$ with $s$-fibre
\begin{equation}
    \TT_{\mu_1,\mu_2}^{+,s} = \{B + C_s \in \TT_{\mu_1,\mu_2} : B \in\n\}
\end{equation}
where $\n\subset\mat(N)$ is the unipotent subalgebra of uppertriangular matrices. 

\acom{or---as Joel pointed out, may be ok with:
The slice $\TT_\mu$ as defined in MVy, no change, \text{and} the family of slices $$\TT_{\mu_1,\mu_2}^{+,s} = \TT_\mu \cap \n + C_{\mu_1,\mu_2}^s$$ where 
% 
\begin{equation}
    C_s \text{ is the block diagonal matrix of companion matrices of } t^{\mu_{1,k}}(t-s)^{\mu_{2,k}}
\end{equation}
% not sure about notation

}

The [family of] orbit[s $\OO_{\lambda_1,\lambda_2}$ fibred over $C = \AA^1$ with $s$-fibre]  
\begin{equation}
    \OO_{\lambda_1,\lambda_2}^s = \{A \in \cN_s : A \text{ is conjugate to } J_{\lambda_1}\oplus (sI_{N_2} + J_{\lambda_2})\}
\end{equation}
where $J_{\lambda_i}$ is the Jordan normal form of block type $\lambda_i$ and $I_{N_2}$ is the identity matrix in $\mat(N_2)$.
% 
\section{Exposition}

In Anderson and Kogan conjectured in \cite{anderson2006algebra} \acom{or earlier?} that those MV polynomials which are cluster monomials for a Fomin--Zelevinsky cluster algebra structure on $\CC[N]$ are naturally expressible as determinants\dots
and they conjecture a formula for many of them.

It's not clear how this work helps/relates to AK/their conjectures.

The generalized MVy is interesting in its own right.

Computing fusion still hard but at least boiled down to linear algebra. Cf.\ fusion product as it appears in BD, FL, MV, AK, BFM.

\begin{quotation}
However, we make crucial use of an idea of Drinfeld, going back to around 1990. He discovered an elegant way of obtaining the commutativity constraint by interpreting the convolution product of sheaves as a ``fusion'' product.
\end{quotation}

Exchange relations only work on cluster modules where one is a mutation of the other (i.e.\ those corresponding to cluster monomials which are related by mutation). Of course this gives me everything. Up to $A_4$ as in type $A_5$ there exist indecomposable modules which are not cluster, so exchange relations do not apply. The hope (conjecture) is that this paper gives a way to compute on such modules. Cf.\ counterexample satisfying $\barD(c_Y) = \barD(b_Z) + 2\barD(b)$, where $b$ is (possibly) cluster, and suggesting that $c_Y = b_Z + 2b$. Can still consider $ext(M_Z)$. Say $b_Z^2 = b_{Z_1} + b_{Z_2}$. 

% No; $x*y=\sum z\Rightarrow y = \frac 1 x \sum z$. 

Representation theory? 

\section{Rising Action} % Problem

\section{Climax}

\section{Falling Action} % Resolution

\begin{theorem}
    Let $\lambda_i\ge\mu_i$ be dominant ($i=1,2$), $\mu = \mu_1 +\mu_2$, and $\lambda =\lambda_1+\lambda_2$. 
    % I have an irrational dislike of ending sentences with subscripts --- can we fix this?
    There is an isomorphism 
    \begin{equation}
        \overline{\OO_{\lambda_1,\lambda_2}}\cap\TT_{\mu_1,\mu_2} \to \overline{(\Grbd[2])^{\lambda_1,\lambda_2}}\cap \cW_{\mu_1,\mu_2}
    \end{equation}
    got by taking a $\mu\times\mu$ block matrix $A$ in the $s$-fibre $\overline{\OO_{\lambda_1,\lambda_2}^s}\cap\TT_{\mu_1,\mu_2}^s$ on the left to the representative of the $s$-fibre on the right defined by  
    \begin{equation}
        \begin{split}
            g &= t^{\mu_1} (t-s)^{\mu_2} + a(t) \\
            a_{ij}(t) &= - \sum_{k=1}^{\mu_i} A^k_{ji} t^{k-1}
        \end{split}
    \end{equation}
    where $A^k_{ji}$ is the $k$th entry from the left of the last row of the $\mu_j\times\mu_i$ block of $A$. 
\end{theorem}

Let's call this the MVyBD isomorphism.

\begin{proof}
    The proof is fibre by fibre, so fix $s\ne 0$. \acom{Emphasize in the intro later (because this always confuses me) that by the $s$-fibre we really mean the $(0,s)$-fibre; i.e.\ its the BD Grassmannian over the second symmetric power of $C = \AA^1$; better just replace $s$-fibre by $(0,s)$-fibre everywhere it occurs.}
    \begin{enumerate}
        \item The map is well defined. In particular, it defines $\CC[t]$-lattices in $\CC(t)^m$. Moreover, these lattices break down to give pairs of lattices upon inverting $t$ or $t-s$ that have the right properties. [Copy Roger's proof]
        \item The inverse map is got by taking the matrix of multiplication by $t$.  More precisely, let $ L \in Gr^{BD} \cap \cW_\mu$.  We work with the quotient $\CC[t]^m/L$ just as in the ordinary MVy isomorphism---the only difference being $\CC\xt$ is replaced by $\CC[t]$.
\begin{enumerate}
    \item 
    We claim that 
    \begin{equation}
        \{[e_i],[te_i],\dots,[t^{\mu_{i}-1}e_i] : 1\le i \le m\}
    \end{equation}
    is a $\CC$-basis of $\CC[t]^m/L$.
    
    To see this, we use that $ L $ 
 has a $\CC[t]$-basis of the form 
    \begin{equation}
        v_i = t^{\mu_i} + \sum_{j>i} p_{ij}(t) e_j 
    \end{equation}
    with $\deg p_{ij}(t) < \mu_i = \mu_{1,i} + \mu_{2,i}$ ($1\le i\le m$).
    \acom{I don't know why this should be true. We might have to just define fibres of $\cW_{\mu_1,\mu_2}$ in this way?}
    \item $t\big|_{\CC[t]^m/L}$ will have two eigenvalues, 0 and $s$, and its generalized 0-eigenspace will have block type $\le \lambda_1$ while its generalized $s$-eigenspace will have block type $\le \lambda_2$. 
    To see this, note that there is a natural isomorphism
    $$\CC[[t]]^m/(L \otimes_{\CC[t]} \CC[[t]]) = \text{generalized $0$ eigenspace of $t$ on } \CC[t]^m/L$$
    carrying the action of $t $ to the action of $t$.
    
    The left hand side is the same thing as
    $$ \CC[[t]]^m / (L \otimes_{\CC[t]} \CC[[t]]) = (\CC[t]^m/L) \otimes_{\CC[t]} \CC[[t]] $$
    
    the defining fact that lattices satisfying Equation~\ref{eq:defGrBDlambda} equivalently satisfy 
    \begin{equation}
        \begin{split}
            t\big|_{\CC\xt^m/L_1}\text{ has Jordan type} \le \lambda_1 \\
            t\big|_{\CC\xt^m/L_2}\text{ has Jordan type} \le \lambda_2 
        \end{split}
    \end{equation}
    where recall 
    $L_i = L\otimes \CC\xt$ ??? 
    % $L_i = L\otimes \CC[(t-p_i)^{-1}]$ 
    and $p_1 = s$ while $p_2 = 0$. \acom{Somehow, restricting to an eigenspace is like inverting/forgetting the action of $t$ by any other generalized eigenvalue? Basic linear algebra? Joel?}
\end{enumerate}
    \end{enumerate}
\end{proof}



\begin{theorem}[Theorem 1 version 2]
    Let $\lambda_1,\lambda_2$ and $\mu$ 
    %be dominant
    be arbitrary, such that $\lambda = \lambda_1 + \lambda_2 \ge \mu$. Then there is an isomorphism 
    \begin{equation}
        \overline{\OO_{\lambda_1,\lambda_2}} \cap \TT_\mu \to \overline{\Grbd[2]^{\lambda_2,\lambda_2}}\cap\cW_\mu 
    \end{equation}
    defined by the same map as in Theorem 1.
\end{theorem}

\jcom{This is true as stated with the ``larger'' definition of $ \cW_\mu $.  In fact, for any $\lambda_1, \lambda_2$, it is true $ \overline{\Grbd[2]^{\lambda_2,\lambda_2}}\cap\cW_\mu $ is contained in a subset that we could call $ \cW_\mu^s$ which we could define as
$$
\cW_\mu^s = G_1[[t^{-1}]]t^\mu \cap G[t,t^{-1}, (t-s)^{-1}] / G[t]
$$
where we regard $G[t,t^{-1}, (t-s)^{-1}] / G[t] \subset G((t^{-1}))/G[t] $

The way to think about this is as follows: inside the thick affine Grassmannian we can consider the $G$-bundles trivialized away from just 0, $s$, or equivalently those lattices which become the standard lattice after tensoring with $ \CC[[t-a]] $ for any $ a \ne 0, s$.
}
\begin{corollary}
    The MVyBD isomorphism restricts to an isomorphism of subfamilies 
    \begin{equation}
        \overline{\OO_{\lambda_1,\lambda_2}}\cap\TT_{\mu_1,\mu_2}^+ \to \overline{(\Grbd[2])^{\lambda_1,\lambda_2}}\cap S_{\mu_1,\mu_2}\,. 
    \end{equation}
\end{corollary}

\begin{proof}
    Let $ A \in \overline{\OO_{\lambda_1,\lambda_2}}\cap\TT_{\mu_1,\mu_2}^+$ and let $ g $ be the polynomial matrix formed by the Mirkovic-Vybornov isomorphism.  Then the diagonal entries of $ g $ are $ t^{\mu_{1,k}} (t-s)^{\mu_{2,k}}$ and we can factor
    $$ g = (g t^{-\mu_1}(t-s)^{-\mu_2}) t^{\mu_1}(t-s)^{\mu_2} \in N[t, t^{-1}, (t-s)^{-1}] t^{\mu_1} (t-s)^{\mu_2}$$
    So we get containment in one direction.
    
    For the reverse containment, we choose $ [g] \in \overline{(\Grbd[2])^{\lambda_1,\lambda_2}}\cap S_{\mu_1,\mu_2}$.  By the lemma below, $[g] \in \cW_\mu$ and thus it lies in the image of our map and we are done.
\end{proof}

% Define $S_{\mu_1,\mu_2}^s = N_-\xT[t^{-1}] t^{\mu_1} (t-s)^{\mu_2}$. 

\acom{is it a fibre of $S_{\mu_1,\mu_2}$ defined above?}

We could also make the following claim. 


\begin{lemma}[KWWY14]
    Let $\mu$ be dominant. Then 
    \begin{equation}
        N_{-}\xT[t^{-1}] L_\mu = N_1\xt[t^{-1}]L_\mu
    \end{equation}
    \acom{where I am not sure about the double brackets.}
\end{lemma}

\begin{lemma}%[Roger's lemma]
    Let $\mu_1,\mu_2$ be dominant and let $s\in \AA^1 - \{0\}$. Then 
    \begin{equation}
        S_{\mu_1,\mu_2}^s \subset \cW_\mu 
    \end{equation}
    where $\mu = \mu_1 + \mu_2$.
\end{lemma}

\begin{proof}
    % Copy Roger's proof.
    We have
\[
\begin{split}
    S_{\mu_1, \mu_2} & = N((t^{-1}))t^{\mu_1}(t-s)^{\mu_2} \\
     & \subset T_1[[t^{-1}]] N((t^{-1})) t^{\mu_1} (t-s)^{\mu_2} \\
     & = T_1[[t^{-1}]] N_1[[t^{-1}]] t^{\mu_1} (t-s)^{\mu_2} \qquad \text{\cite[Lemma 2.3]{kamnitzer2014yangians}}\\
     & = B_1[[t^{-1}]] t^{\mu_1} (t-s)^{\mu_2} \\
     & = B_1[[t^{-1}]] t^{\mu_1 + \mu_2} \\
     & \subset G_1[[t^{-1}]] t^{\mu_1 + \mu_2} \\
     & = W_{\mu_1 + \mu_2}
\end{split}
\]
where $B_1[[t^{-1}]] t^{\mu_1} (t-s)^{\mu_2} = B_1[[t^{-1}]] t^{\mu_1 + \mu_2}$ since 
\[
\frac{t}{t-s} = 
1 + \frac{s}{t} + \frac{s^2}{t^2} + \cdots 
\in B_1[[t^{-1}]].
\]
\end{proof}

\section{Denouement}

As an application we can compute fusion of stable MV cycles of type $\alpha_i$ for any $i \in I$. What about more general weights? Having $\kpf > 1$.

\begin{proposition}
    % Given two MV cycles $Z_\tau$ and $Z_\sigma$ of type\dots 
    Let $Z_{i} = Z_{\tau_i}$  be an MV cycle of type $\lambda_i\in P_+$ and weight $\mu_i\in P$ ($i = 1,2$). Then 
    \begin{equation}
        % Z_\tau\ast Z_\sigma 
        m_{\lambda_1\lambda_2} ([Z_1]\otimes[Z_2]) = \sum_{Z\in\cZ(\lambda)_\mu} i\left(
            Z,\pi^{-1}(0)\cdot\overline{Z_1\times Z_2\times U}
        \right) [Z]
    \end{equation}
    is found by\dots 
    In particular, the multiplicity of $Z_\tau$ in the product of the zero fiber 
    % (divisor class) 
    and the 
    % (Zariski) 
    closure of the family $Z_1 \times Z_2 \times U$ is equal to the multiplicity of $X_\tau$ among 
    \begin{equation}\label{eq:tabmult}
        \irr \lim_{s\to 0} \overline\OO^s_{\lambda_1,\lambda_2} \cap \TT_{\mu_1,\mu_2}^{+,s}
    \end{equation}
    and since stable MV cycles can be represented by ordinary ones the multiplicities in 
    $$
    b_{Z_1}b_{Z_2} = \sum_{Z\in\cZ(\infty)_{-\nu_1 - \nu_2}} i \left(
        Z, \pi^{-1}(0) \cdot \overline{Z_1 \times Z_2 \times U}
    \right) b_Z 
    $$
    can also be deduced from \cref{eq:tabmult} for appropriate choices of $\lambda_i$ and $\mu_i$ ($i = 1,2$).
\end{proposition}
\acom{or Corollary?}

\begin{conjecture}
    % Let $Z_i \subset \overline{S^{\nu_i}\cap S^0_-}$ be an MV cycle of weight $\nu_i$ ($i = 1,2$) and put $\nu = \nu_1 + \nu_2$. 
    Let $\tau$ be the tableau of shape $\lambda$ and weight $\mu$ whose Lusztig datum is equal to the sum of the Lusztig data of $\tau_1$ and $\tau_2$. 
    % Then $i(\tau, \pi^{-1}(0) \cdot \overline{\tau_1 \times \tau_2 \times U})$ is equal to 1. 
    Then 
    \begin{equation}
        i(\tau, \pi^{-1}(0) \cdot \overline{\tau_1 \times \tau_2 \times U}) = 1 \,. 
    \end{equation}
\end{conjecture}

\bibliographystyle{alpha}
\bibliography{mvybd}

\end{document}