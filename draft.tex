%!TEX program = lualatex

\documentclass[draft]{article}
\usepackage{basic}
\usepackage{emoji}

% \setemojifont{Apple Color Emoji}
% Working title: 
\title{How to compute the fusion product of MV cycles in type A}
\author{Roger Bai, Anne Dranowski, Joel Kamnitzer} 
\date{October 2020}

\begin{document}

\maketitle

\section{Introduction}
% 
The Mirkovic--Vybornov isomorphism (\cite{mirkovic2007quiver,mirkovic2019comparison}) relates slices in the affine Grassmannian to slices in conjugacy classes of nilpotent matrices and Nakajima quiver varieties. In so doing it provides a geometric version of symmetric and skew Howe $(\GL_m,\GL_n)$ dualities. 
% 

[Liaison]

The Beilinson--Drinfeld Grassmannian, essential to \cite{mirkovic2007geometric}, is a space which is used to define a fusion (aka convolution) product on sheaves (representations) as well as MV cycles (bases). 
% 
This product structure makes the vector space spanned by the MV cycles into a commutative algebra isomorphic to the ring of functions on the unipotent subgroup (conjectured in \cite{anderson2003polytope} and proved in \cite{baumann2019mirkovic}). % conjectured by JEA and proved by BKK
% 
To quote from Mirkovic and Vilonen:
\begin{quotation}
    However, we make crucial use of an idea of Drinfeld, going back to around 1990. He discovered an elegant way of obtaining the commutativity constraint by interpreting the convolution product of sheaves as a ``fusion'' product.
\end{quotation}
% 
[Liaison]

The purpose of this paper is to give an elementary way to compute this product in type $A$. We do not know how fruitful this work might be in the larger landscape of geometric representation theory as it is plagued by two major limitations: it does not readily generalize outside of type $A$, and it is as yet computationally expensive. 
% 
Moreover, it is not clear why one should care about the fusion product on bases. \emoji{imp} \emoji{grin} \emoji{worried} \emoji{monocle-face} \emoji{skull}

\section{Players}

Fix $G = \GL_m$ and let $T\subset G$ be the maximal torus of diagonal matrices. 
% are there other possibilities?? :S 
Write $P$ for the coweight lattice of $T$, so we have $P\cong \ZZ^m$.  The dominant coweights of $ G $ is the set of $ \lambda = (\lambda_1, \dots, \lambda_m) $ such that $ \lambda_1 \ge \cdots \ge \lambda_m$.  We also need the notion of effective dominant coweight, 
% \acom{can we conflate or should we be precise and say coweight?} \jcom{Corrected, let's stick to coweights.}
which means those $ \lambda $ as above with $ \lambda_m \ge 0$.  So an effective dominant coweight $ \lambda $ is a partition of $ \lambda_1 + \cdots + \lambda_m$, which we call the size of $ \lambda$. Write $P_+$ for dominant and $P_{++}$ for effective dominant. 


Set $\cO = \CC\xt$ and $\cK = \CC\xT$. We will also need the notation $\cO_s = \CC\xt[t-s]$ for the completion of the local ring $\CC[t]$ at $s\in\spec\CC$ and $\cK_s$ for its fraction field $\CC\xT[t-s]$, as well as $\Kinf = \CC\xT[t^{-1}]$ \acom{for the fraction field of the completion of $\CC[t]$ at $\infty$?} 
% TODO: given by or described by?
% 
Note that (by changing the local coordinate on $\spec\CC[t]$ to $t-s$) we can identify $\cO_s \cong \cO$ and $\cK_s\cong\cK$. 

Given a coweight $\mu \in P$ and $ s\in \CC$, we define $ (t-s)^\mu$ to be the diagonal matrix 
% with entries $ (t-s)^{\mu_1}, \dots, (t-s)^{\mu_m}$
% \[
%     \begin{pNiceMatrix}[nullify-dots,xdots/line-style=loosely dotted]
%         (t-s)^{\mu_1} & 0             & \Cdots & 0 \\
%         0             & (t-s)^{\mu_2} & \Ddots & \Vdots \\
%         % 0 & b & a & \Ddots & & \\
%         \Vdots              & \Ddots        & \Ddots & 0 \\
%         % \Vdots & & & & \\
%         0             & \Cdots        &   0    & (t-s)^{\mu_m}
%     \end{pNiceMatrix}
% \]
\[
\begin{bmatrix}
    (t-s)^{\mu_1} \\
    & (t-s)^{\mu_2} \\ 
    & & \ddots \\
    & & & (t-s)^{\mu_m}
\end{bmatrix} 
\]
which we will view in $\GL_m(R) $ for any ring $ R $ containing $\CC[t]$ in which $t-s$ is invertible (for example $ R = \Ks, \CC(t)$).

We will also study the affine space of $m\times m$ matrices which we denote $ M_m$.
% nicematrix test
% $\begin{bNiceArray}{ccc|c}[margin]
%     \Block{3-3}<\Large>{A} & & & 0 \\
%     & \hspace*{1cm} & & \Vdots \\
%     & & & 0 \\
%     \hline
%     0 & \Cdots& 0 & 0
%     \end{bNiceArray}$
% 
% $\begin{pNiceMatrix}
%     \frac12 & -\frac12 \\
%     \frac13 & \frac14 \\
%     \end{pNiceMatrix}$

% Set $\AA = \AA^1_x$ (the subscript denoting the chosen local coordinate) and fix $s\in \AA - \{0\}$. 

\subsection{Lattices}
\label{ss:lat}
We will use the lattice model, so it is useful to recall the following definition.  Let $ R \subset K$ be two rings (usually, but not always, $K$ will be a field).  Let $ m \in \mathbb N$ and consider $ K^m $ as a $K$-module.   By restriction $ K^m$ can be viewed as an $R$-module.  An $R$-lattice in $M$ is an $R$-submodule $ L \subset K^m$ such that $ L $ is a free rank $ m $ $R$-module and $ L \otimes_R K = K^m $.  Equivalently, $ L = \Sp_R(v_1, \dots, v_m)$ where $v_1, \dots, v_m$ are free generators of $K^m$. 

% The trivial lattice is 
$L_0(R) = R^m \subset K^m $ is called the standard lattice. The group $\GL_m(K) $ acts transitively on the set of $R$-lattices in $K^m$, thus giving a bijection between 
% the set of such lattices 
this set and $\GL_m(K)/\GL_m(R)$.

We will be particularly interested in $\CC[t]$-lattices in $ \CC(t)^m$.  Given such a lattice $ L $ and a point $ a \in \CC$, the specialization of $ L $ at $ a $ is defined as
$$
L(a) := L \otimes_{\CC[t]} \CC\xt[t-a] \subset \CC\xT[t-a]\,.
$$
We say that $ L $ is trivial at $ a $ if $ L(a) = \CC\xt[t-a]^m$. For example, the lattice $(t-s)^{-1}\CC[t] \subset \CC(t)$ is trivial at any $a\ne s$. 

% \acom{Should we also mention our interest in $\cO$-lattices in $\cK^m$?}
% \jcom{I don't think that it is necessary.  I just put the part about $ \CC(t)$ to set up the notation for this specialization.}


% \jcom{I wasn't sure of the notation to use here, but I wanted something which didn't conflict with $ L_0$.}

% \acom{I think that MVy use this notation also: }

%\includegraphics[width=\textwidth]{img/Capture d’écran, le 2021-02-17 à 13.03.10.png}}

\subsection{Affine Grassmannians} %  actors

Set $\AA = \AA^1$ and $\PP = \PP^1$. 

\begin{itemize}
    \item The ordinary affine Grassmannian $\Gr = G(\cK)/G(\cO)$. It is described in modular terms by 
    $$
    \Gr = \{ (P, \varphi) : \text{$V$ is a rank $m$ vector bundle on $ D_0 $, $\varphi : V \rightarrow V_0 $ on $ D_0 \setminus \{0\} $} \} 
    $$
    where $V_0$ is the trivial rank $m$ vector bundle, and $ D_0$ is the formal disk centered at $ 0 \in \PP$.
    
    We also have a lattice description 
    $$ 
    \Gr = \{ L \subset \cK^m : \text{ $L$ is a $\cO$-lattice} \}\,.
    $$
    % 
    \item For any $ s \in \CC $, \acom{should it be $\AA$ instead of $\CC$} the ordinary affine Grassmannian $\Gr_s = G(\cK_s)/G(\cO_s)$ at $ s $. It is described in modular terms by
    $$
    \Gr_s = \{ (P, \varphi) : \text{$V$ is a rank $m$ vector bundle on $ D_s $, $\varphi : V \rightarrow V_0 $ on $ D_s \setminus \{s\} $} \} 
    $$
    where $D_s$ is the formal disk centered at $s\in\PP$.

    We also have a lattice description 
    $$ 
    \Gr_s = \{ L \subset \cK_s^m : \text{ $L$ is a $\cO_s$-lattice} \} \,. 
    $$   
    \item The thick affine Grassmannian $\Grth = G(\Kinf)/G(\CC[t])$. % \CC\xT[t^{-1}]
    It is described in modular terms by
    $$
        \Grth = \{ (P, \varphi) : \text{$V$ is a rank $m$ vector bundle on $ \PP $, $\varphi : V \rightarrow V_0 $ on $ D_\infty $ } \} % (a disk centered at $ \infty$)
    $$
    where $D_\infty$ denotes a disk centered at $ \infty$. 
    % \acom{What is the difference between ``a disk'' and ``the formal disk''} \jcom{No difference.}
    
    We also have a lattice description 
    $$ 
    \Grth = \{ L \subset  \Kinf^m : \text{ $L$ is a $\CC[t]$-lattice} \}\,.
    $$ % \CC((t^{-1}))^m 
    
    \jcom{Maybe write $\Kinf$ for $ \CC\xT[t^{-1}] $?  Also maybe change the ``thick'' notation''.} \acom{Since $\sf th$ is also the first two letters of ``thin''?} \jcom{That's a good point!  Maybe we could use some kind of bold  Gr (``thick'' letters) or maybe that would be too confusing.} \acom{Ok, it's tentative.}

    \item The (two point) Beilinson--Drinfeld Grassmannian $\pi : \Grbd\to \AA$ (with one point fixed at 0, and the second point $s\in\AA$ varying).
    It is described in modular terms by
    % {\small
    $$
    \Grbd = 
    % \begin{aligned}
        \left\{ 
            (V,\varphi,s) : V\text{ is a rank $m$ vector bundle on }\PP, \varphi : V \rightarrow V_0 \text{ on } \PP \setminus \{0, s\}  
        \right\} 
    % \end{aligned}
    $$%}

    The fibre of $\Grbd \to \AA$ over $ s \in \AA $, for $ s \ne 0$, will be denoted $ \Gr_{0,s} $ and is given by
    % 
    $$ 
    \Gr_{0,s} := \pi^{-1}(s) = G(\CC[t, t^{-1}, (t-s)^{-1}]/G(\CC[t])\,.
    $$
    % \acom{0 in notation is kinda redundant? But we can't take it away because we already have a $\Gr_s$ and it is something else?}
    % \jcom{Yes, that is one reason.  But also, there was a comment of yours before in the file, saying that you sometimes got confused that one point was fixed at 0, so this is a good notation to remind us (and the reader).  For the same reason, maybe we should write $ \Gr_{0, \AA}$ instead of $ \Gr_\AA$.}
    % \acom{lol}
    
    We also have a lattice descriptions 
    % {\small
    \begin{gather*}
    \Grbd = 
    \{ L \subset  \CC(t)^m, s \in \CC : \text{ $L$ is a $\CC[t]$-lattice 
    % , $L$ is 
    trivial at any $ a \ne 0, s$} \} \\
     \Gr_{0,s} = 
    \{ L \subset  \CC[t,t^{-1},(t-s)^{-1}]^m : \text{ $L$ is a $\CC[t]$-lattice} \}
    \end{gather*}%}
    \acom{$s\in\CC$ should be $s\in\AA$?}
    % 
    \item The positive parts $ \Gr^+$ and ${\Grth}^+$ of the affine Grassmannian and of the thick affine Grassmannian resp.\ are defined as 
    % and of the thick affine Grassmannian ${\Grth}^+$ is defined as
    $$
    \Gr^+ = (M_m(\cO) \cap G(\cK)) / G(\cO) \quad {\Grth}^+ = (M_m(\CC[t]) \cap G(\CC\xT[t^{-1}]) / G(\CC[t]) 
    $$
    % \acom{What is $M_m$?} \jcom{The space of matrices.  I added it to the beginning of section 2.}

    In modular terms, $\Gr^+$ (resp. ${\Grth}^+$) is the set of those $ (V, \varphi)$ where $ \varphi : V \rightarrow V_0 $ extends to an inclusion of coherent sheaves over $ D_0 $ (resp. over $ \PP$).  
    
    In lattice terms, $ \Gr^+$ (resp. ${\Grth}^+$) contains those lattices $L$ which are contained in the standard 
    % \acom{previously called ``trivial''} \jcom{I changed it so that $ R^m $ is called the standard lattice.  But I would like to leave the terminology ``trivial at $ a$'' is that ok, or should we write ``standard at $a$''.} 
    % \acom{Oh, I don't know.. is trivialized at $a$? is standardized at $a$? Maybe standard at $a$ is better.. Man I never thought about this interchangeable lingo.} 
    lattice $ L_0(\cO)$ (resp. $L_0(\CC[t])$. % , i.e. $ L \subseteq L_0$.
\end{itemize}

These \acom{which?} different versions of the affine Grassmannian are related as follows.  

First there is an isomorphism $ \Gr_s \cong \Gr $ coming from the isomorphism $ \cK \cong \Ks$.

Second, we have a map from the two point BD affine Grassmannian to the thick affine Grassmannian $ \Grbd \rightarrow \Grth $
% $$ \Grbd \rightarrow \Grth $$ 
which is given in modular terms by restricting the trivialization to $ D_\infty$.  It is given in group-theoretic terms on the fibre over $ s $ by
$$
G(\CC[t, t^{-1}, (t-s)^{-1}])/ G(\CC[t]) \rightarrow G(\Kinf)/G(\CC[t])
$$
% \acom{Permission to update notation and replace $\CC\xT[t^{-1}]$ by $\Kinf$ as per your suggestion Joel?}
% \jcom{Permission granted!}
% \acom{replaced}

Finally, it is given in lattice terms as the identity on the lattice and using the inclusion $\CC[t, t^{-1}, (t-s)^{-1}]^m \rightarrow \Kinf^m$ on the ambient spaces.

The fibres $ \Gr_{0,s}$ of $ \Grbd \rightarrow \AA$ can be described as % are given as follows
$$
\Gr_{0,s} \cong 
\begin{cases} 
    \Gr \times \Gr_s & s \ne 0 \\
    \Gr              & s = 0\,.
\end{cases}
$$
In the $s\ne 0$ case, the isomorphism is given in the modular realization by restricting the vector bundle and trivializating to the appropriate disks.  
In the lattice realization, it is given by forming tensor products 
$$
L \mapsto (L(0), L(s))\,.
$$
% in the $ s \ne 0 $ case.
\jcom{Maybe we should introduce some notation for this isomorphism.}
% \acom{Ok. Question: how is the isomorphism given at $s=0$?}
\acom{Ok. TODO.}

In the $ s = 0 $ case, the isomorphism is described in the same way, except that we just need to form $ L(0)$.

\jcom{I tried to answer your question, but didn't explain it well.  When $ s = 0$, the map is the same except that $ L(0) = L(s) $ , so we only need $L(0)$.}


\subsection{The fusion construction}
The following construction will be very important in this paper.  Let $ X_1, X_2 \subset \Gr$ be two subschemes. Then using the isomorphism $ \Gr \cong \Gr_s $, for each $ s $, we obtain a subscheme $ X_1 \times X_2 \subset \Gr \times \Gr_s $.  
% <<<<<<< HEAD
% Via \ul{this} \acom{rather, the above? $L\mapsto (L(0),L(s))$} 
Next, using the isomorphism $ \Gr_{0,s} \cong \Gr \times \Gr_s $,  for each $ s \ne 0$, we obtain a subscheme $ X_1 \times X_2 \times \AA^\times $  of $\Grbd$.  
% =======
% Via the isomorphism $ \Gr_{0,s} \cong \Gr \times \Gr_s $,  for each $ s \ne 0$, we obtain a subscheme $ X_1 \times X_2 \times \AA^\times $  of $\Grbd$.  
% >>>>>>> 5a15ce15b14109bab482282aa5657c9596f7892d

We define $ X_1 \ast_{\AA} X_2 $ to be the scheme-theoretic closure of $ X_1 \times X_2 \times \AA^\times $ inside of $ \Grbd $.  
We use the projection $ \pi: \Grbd \rightarrow \AA $ to form $ X_1 \circ X_2 = X_1 \ast_\AA X_2 \cap \pi^{-1}(0) = \Gr_{0,0} $, which we call the fusion of $ X_1 $ and $X_2$.  
We regard $ X_1 \circ X_2$ as a subscheme of $ \Gr $ using {the isomorphism} $\Gr_{0,0} \cong \Gr $. 
% \acom{Repeat question: what is this iso? Roger says maybe it's in MV?}
% \jcom{Now answered above.}

\jcom{I thought that it would be good to introduce this operation now, since we will use it often in the paper and it is the most convenient way to define $\Gr^{\lambda_1, \lambda_2}$. I don't know what notation we should use.}

\acom{Roger and I have been using $\ast$ to denote fusion. Do we need a notation for the scheme-theoretic closure of $X_1\times X_2\times\AA^\times$? If not then I like $X_1\ast X_2 = \overline{X_1 \times X_2 \times \AA^\times}\cap \pi^{-1}(0)$.}
\jcom{I think that we will need notations for both things, but I'm happy to use $ \ast$ for the central fibre and use something else for the whole family.} \acom{Ok, I have changed it to notation from Roger's thesis/AK. Only now I am confused as to what is being called fusion---why do we not take the top dimensional subscheme?}

\subsection{Some subvarieties of affine Grassmannians}
Now fix effective dominant weights $ \lambda_i,\mu_i \in P_{++}$ of size $N_i$ ($i=1,2$) and their sums $ \lambda = \lambda_1 + \lambda_2, \mu = \mu_1 + \mu_2$ of size $ N = N_1 + N_2$.

% 
\begin{itemize}
    % \item Weights $\mu_i\le\lambda_i$ of size $N_i$ ($i=1,2$) 
    % \item Their sums $\mu = \sum \mu_i \le \lambda = \sum\lambda_i$ of size $N = \sum N_i$
    \item The point $L_\lambda = t^\lambda G(\cO)$ in $\Gr$. In the lattice model, we have
    $$ 
    L_\lambda = \Sp_\cO(t^{\lambda_1}e_1, \dots, t^{\lambda_m}e_m)\,. 
    $$
    \item The orbit $\Gr^\lambda = G(\cO) L_\lambda$ and its closure 
    $ \overline\Gr^\lambda = \bigcup_{\lambda' \le \lambda} \Gr^{\lambda'} $.  
    \acom{do we prefer $\overline{\Gr}^\lambda$ to $\overline{\Gr^\lambda}$? I am for the former! Though it is less accurate??} \jcom{I'm not sure which is better.  I was ``testing'' it out.}
    This orbit lies inside the positive part $ \Gr^+$ of the affine Grassmannian.
    % 
    \item The family $ \overline{\Gr}^{\lambda_1, \lambda_2} \rightarrow \AA$ defined by $ \overline{\Gr}^{\lambda_1, \lambda_2} = \overline{\Gr}^{\lambda_1} \ast_\AA \overline{\Gr}^{\lambda_2}$. 
    % \acom{I guess that it is good to have the notation here, in place of $\overline{\overline\Gr^{\lambda_1}\times\overline\Gr^{\lambda_2}\times\AA^\times}$..}
    \acom{should we also subsript this family with $0,\AA$?}
    
    By construction the fibre $ \overline{\Gr}_{0,s}^{\lambda_1, \lambda_2} $ over $ s \ne 0 $ is $ \overline{\Gr}^{\lambda_1} \times \overline{\Gr}^{\lambda_2}$.  By Zhu's theorem (\jcom{need ref} \acom{pretty sure it's \cite[Proposition 3.1.14]{zhu2016introduction}}), the fibre over $ 0 $ is $ \overline{\Gr}^{\lambda_1 + \lambda_2}$.
    
    Suppose that $ s \ne 0$.  In the fibre $\overline{\Gr}_{0,s}^{\lambda_1, \lambda_2}$ we have the open locus $ \Gr_{0,s}^{\lambda_1, \lambda_2}$ coming from transporting $ \Gr^{\lambda_1} \times \Gr^{\lambda_2} $ via the isomorphism $ \Gr_{0,s} \cong \Gr \times \Gr_s $.  This open locus is a $ G(\CC[t])$-orbit
    % .  The lattice description of $ \Gr_{0,s}^{\lambda_1, \lambda_2}$ 
    whose lattice description is given in \Cref{le:Grl1l2} below.
    
    \item The locus $\cW_\mu = G_1\xt[t^{-1}]L_\mu \subset \Grth $.  
    \acom{want $t^\mu\CC[t]$ as opposed to $L_\mu$ here?}
    In modular terms, this corresponds to the locus of those $ (V, \varphi)$ such that $ V $ is isomorphic to the trivial vector bundle on $ \PP$ and such that $ \varphi$ preserves the \ul{Harder--Narasimhan filtration of $V$ at $ \infty$}.  The lattice description of $ \cW_\mu \cap \Grth^+$ is given in Lemma \ref{le:Wmu} below. \acom{que est-ce?}
    
    \item The semi-infinite orbit $ S^\mu = N(\cK)L_\mu \subset \Gr $.  
    \item The family of semi-infinite orbits $ S^{\mu_1, \mu_2}_{0,\AA} := S^{\mu_1} \ast_\AA S^{\mu_2}$.  If $ s \ne 0 $, then the fibre $S^{\mu_1, \mu_2}_{0,s} $ is given by
    $$
    S^{\mu_1, \mu_2}_{0,s} = N(\CC[t, t^{-1}, (t-s)^{-1}])t^{\mu_1} (t-s)^{\mu_2} \,. % \G(\CC[t])
    $$
    \acom{should be $G(\cO)$-coset or $G(\CC[t])$-coset?}
    On the other hand, if $ s = 0$, then the fibre $S^{\mu_1, \mu_2}_{0,0} $ equals $ S^\mu$.
    
    \jcom{These two statements here require some proof, so perhaps they should be split off as a lemma in ``Rising action''.}
    % TODO!
    
\end{itemize}

\subsection{Matrices}
We now consider some subvarieties of the space of $ N\times N$ matrices. 

%Recall that $ \lambda$ is a partition of $N$ and $ \lambda_1, \lambda_2 $ are partitions of $ N_1, N_2$ with $N_1 + N_2 = N$. 


\begin{itemize}
    \item For $ s \in \CC$, we have a Jordan form matrix $ J_{s,\lambda}$ with eigenvalue $ s$ and Jordan blocks of sizes $ \lambda_1, \dots, \lambda_m$. \acom{$s\in\AA$?}
    % 
    \item The adjoint orbit $ \OO^\lambda \subset M_N(\CC)$ of matrices conjugate to $ J_{0,\lambda}$.  Its closure is $ \overline{\OO}^\lambda = \bigcup_{\lambda' \le \lambda} \OO^{\lambda'}$.
    % 
    \item For $ s \in \CC, s \ne 0$, the adjoint orbit $ \OO^{\lambda_1, \lambda_2}_{0,s}$ of matrices conjugate to $ J_{0,\lambda_1} \oplus J_{s,\lambda_2}$.  Its closure is $ \overline{\OO}^{\lambda_1, \lambda_2}_{0,s} = \bigcup_{\lambda_1' \le \lambda_1, \lambda'_2 \le \lambda_2} \OO^{\lambda_1', \lambda'_2}_{0,s}$. \acom{$s\in\AA$?}
    % 
    \item A linear operator $ T : V \rightarrow V $ on an $N$-dimensional vector space $V$ is said to have Jordan type $((0,\lambda_1), (s,\lambda_2))$ if its matrix representatives lie in $ \OO^{\lambda_1, \lambda_2}_{0,s}$.
    % 
    \item The family of adjoint orbits $ \overline{\OO}^{\lambda_1, \lambda_2}_{0,\AA} \rightarrow \AA$ whose fibre over $ s \ne 0 $ is $ \overline{\OO}^{\lambda_1, \lambda_2}_{0,s}$ and whose fibre over 0 is $\overline{\OO}^\lambda$.  \jcom{We should justify why the 0 fibre is correct scheme-theoretically.} 
    \acom{does it follow because ``$s$ divides $\det g$'' is a closed condition 
    in $\AA_s\times \AA_{\det}$? 
    so the set of $A$ which are conjugate to
    $J_{0,\lambda_1} \oplus J_{s,\lambda_2}$
    by a matrix whose determinant is divisible by $s$ is also closed?}
    
    \item Fix the ``\mu-numeration'' \((e^1_1,\ldots,e^{\mu_1}_1,\ldots,e^1_m,\ldots,e^{\mu_m}_m)\) of the standard basis of $\CC^N$. The Mirkovic--Vybnornov slice $\TT_\mu$ is defined as the set of $A\in M_N(\CC)$ such that 
    \[
        \begin{aligned}
            &\text{for all } 1 \le a,s\le m\,,
            \text{for all } 1\le b\le \mu_a\,, 1\le t\le \mu_s\,, \\
            &\text{if } 1\le t < \mu_s \text{ or } t = \mu_s < b \le \mu_a \\
            &\text{then } (e^t_s)' (A-J_\mu) e^b_a = 0 \,.
        \end{aligned}    
    \]
    In words, $A$ is a $\mu\times\mu$ block matrix with possibly nonzero entries occuring in the first $\min(\mu_i,\mu_j)$ columns of the last row of each $\mu_i\times\mu_j$ block. 
    Mirkovic and Vybornov describe $\TT_\mu$ as the slice in which $f$ acts as close to regular nilpotent as possible.
    % $\dim \TT_\mu = N^2 - \dim \OO_\mu$. 
    \acom{this ok?} % \jcom{Anne, can you add this definition?}
    
    \item To each $ A \in \TT_\mu$, we will associate an $m$\times$m$ matrix of polynomials $ g(A) = \left( g(A)_{ij} \right) \in M_m(\CC[t]) $ as follows.
       \begin{equation}
            g(A)_{ij} = \begin{cases} t^{\mu_i} - \sum_{k=1}^{\mu_i} A^k_{ji} t^{k-1} \text{ if $ i = j$} \\
             - \sum_{k=1}^{\mu_i} A^k_{ji} t^{k-1} \text{ if $ i \ne j$ }
        \end{cases}
    \end{equation}
    where $A^k_{ji}$ is the $k$th entry from the left of the last row of the $\mu_j\times\mu_i$ block of $A$. 
\jcom{I moved the definition here since it will be useful for defining the upper triangular stuff.}
    
    \item The ``upper-triangular'' MVy slice $\TT^{\mu_1, \mu_2,+}_\AA \rightarrow \AA $, defined by
    $$
    \TT^{\mu_1, \mu_2,+}_\AA := \{ (A,s) \in \TT_\mu \times \AA : g(A)_{ii} = t^{\mu_{1,i}} (t-s)^{\mu_{2,i}}, g(A)_{ij} = 0 \text{ for $ j < i $ }\}
    $$
    So a matrix in $\TT^{\mu_1, \mu_2,+}_\AA  $ is weakly block upper-triangular and its diagonal blocks are given by the companion matrices for the polynomials $t^{\mu_{1,i}} (t-s)^{\mu_{2,i}}$.
    
    Note that the fibre $  \TT^{\mu_1, \mu_2,+}_{0,0}$ is the same as the intersection of $ T^\mu $ with the strictly upper triangular matrices.
    
    \jcom{Instead of $\TT^+$ maybe a different letter would be best? I'm moved all $\lambda, \mu$ to upper indices for consistency (except for $ \cW_\mu, \TT_\mu$) and also to let us put $ s $ or $ \AA$ in the lower index.  I haven't been very consistent about $ \AA $ vs $ \CC$.  Which do you prefer?}

    \acom{Maybe $\UU$ for a different letter? And, I guess we do not need $\AA$ but it is a nice letter to have around. It also serves to distinguish the $\CC$ over which $R,K$ are defined from the $\AA$ over which $\Grbd$ is defined?}
\end{itemize}

\subsection{Old stuff}
\jcom{I'm going to delete everything below here in this section, if that is ok with you.}
\acom{Yup!}
% New (?) definitions among these are as follows.
% 
%The [family of] slices [with $s$-fibre?]
%\begin{equation}
%    \cW_{\mu_1,\mu_2} = G_1 [t^{-1},(t-s)^{-1}]L_{\mu_1,\mu_2}
%\end{equation}

Recall (citation?) that there is an embedding of the usual (``thin'') affine Grassmannian $\Gr $ as we have defined it into the ``thick'' affine Grassmannian. 
% 

\acom{In \cite{kamnitzer2014yangians} the thick Gr is said to be necessary for the noncommutative algebras considered. What is our reason for preferring it to the thin Gr? Can we replace thick by thin? Joel said that rather it is needed/used because unlike the thin $\Gr$ which is an ind-scheme, the thick $\Gr^{\sf th}$ is a scheme, so taking Spec of global functions (on $\cW_\mu$?) will recover $\cW_\mu$?}
\[
\Gr \cong G(\CC[t,t^{-1}])/G(\CC[t])\hookrightarrow \Gr^{\sf th} := G(\CC\xT[t^{-1}])/G(\CC[t]) \quad g\cO \mapsto gG(\CC[t])  \,.
\]
Let 
% $L_\mu = t^\mu G(\cO)$ and 
$L_{\mu_1, \mu_2} = t^{\mu_1} (t-s)^{\mu_2} G(\CC[t])$. 
% \acom{or $ t^{\mu_1} (t-s)^{\mu_2} G[t]$ in thick?} % \in G((t^{-1}))/G[t]
% 
% \begin{equation}
%     \cW_\mu = G_1[[t^{-1}]]L_\mu \subset G((t^{-1}))/G[t]
% \end{equation}
% a subscheme of the thick affine Grassmannian
% 
Note that $ L_{\mu_1, \mu_2} = (t-s)^{\mu_2} t^{-\mu_2} L_{\mu} \in \cW_\mu$ 
% \acom{thick or thin}
and that  $L_{\mu_1,\mu_2}\in\Grbd$ \acom{should say $\Grbd[s]$ as it's really a lattice in a given fibre being characterized?} is a $\CC[t]$-lattice in $\CC(t)^m$ that specializes to a $\CC\xt$-lattice in $\CC\xT^m$ away from $t = 0$ and away from $t = s$, i.e.\
% \acom{thick only} 
\begin{equation}
    \label{eq:defWBD}
    % \begin{split}
        %L_{\mu_1,\mu_2}\otimes \CC[(t-s)^{-1}] =
        L_{\mu_1,\mu_2}\otimes \CC\xt = L_{\mu_2} \otimes \CC\xt \qquad
        L_{\mu_1,\mu_2}\otimes \CC\xt[t-s] = L_{\mu_1} \otimes \CC\xt[t-s]
    % \end{split}
\end{equation}
where 
% $L_{\mu_i}$ denotes the point $t^{\mu_i}G(\cO)\in\Gr$ and 
the tensor product is taken over $\CC[t]$. 
% 

The family of semi-infinite orbits 
\begin{equation}
    S_{\mu_1,\mu_2} = N_-(\CC\xT[t^{-1}]) L_{\mu_1,\mu_2}\,. 
\end{equation}
\acom{1. Introduce $N_\pm$. 2. In \cite{kamnitzer2014yangians} the global version of $S^\mu_-$ is the subvariety $S^\mu_{-,\AA}\subset\Grbd$ whose generic fibre is $\bigcup_{\mu_1 + \mu_2 = \mu} S^{\mu_1}_-\times S^{\mu_2}_-$ and whose zero fibre is $S^\mu_-$ which don't look to be the same thing.}

The family \acom{orbit? orbits? family of orbits?} of $\CC[t]$-lattices in $\CC(t)^m$ whose elements specialize again to $\CC\xt$-lattices in $\CC\xT^m$ away from $t = 0$ and away from $t = s$ (i.e.\ which are trivial at $a \ne 0$ or $s$, to use the terminology introduced in the \cref{ss:lat}) is described more explicitly as follows
\begin{equation}
    \label{eq:defGrBDlambda}
    \begin{split}
        % (\Grbd[2])^{\lambda_1,\lambda_2} = 
        \Gr_{\AA}^{\lambda_1,\lambda_2} = 
        \{L \in\Grbd : 
        %L\otimes\CC[t^{-1}] 
        L \otimes_{\CC[t]} \CC\xt[t-s] 
        &\in \Gr^{\lambda_2} \text{ and } \\
        % L\otimes\CC[(s-t)^{-1}] 
        L\otimes_{\CC[t]} \CC\xt &\in \Gr^{\lambda_1} \text{ and } \\ 
        L \otimes_{\CC[t]} \CC\xt[t-a] &= L_0 \text{ for all $ a \ne 0, s$} \}\,.
    \end{split}
\end{equation}

% \jcom{It is important to add the condition that the lattices are trivial at $ a $.} \acom{$a\ne 0,s$}
% 
% \acom{These defining conditions are again just telling us that in the fibre over the fixed point $(0,s)\in C^{(2)}$ this set is pairs of lattices; note that we invert the indeterminates $t$ and $(t-s)$ \textit{after} specializing $(0,s)$. Could we give a more explicit characterization like $t^{N_1}(t-s)^{N_2} L_0 \subseteq L\subseteq t^{-N_1}(t-s)^{-N_2}$?}

The [family ``$\cN\otimes\CC[s]$? $\cN\times\CC$?'' of] semi-nilpotent cone[s fibred over $C = \AA^1$ with $s$-fibre] 
\begin{equation}
    \cN_s = \{A \in \mat(N) : \text{eigenvalues of } A \text{ are } 0 \text{ or }s\}\,.
\end{equation}

\begin{comment}
    The [family of] slice[s $\TT_{\mu_1,\mu_2}$ fibred over $C=\AA^1$ with $s$-fibre]
\begin{equation}
    \begin{split}
        \TT_{\mu_1,\mu_2}^s = \{ B + C_s: B &\text{ is a }\mu\times\mu \text{ block matrix of zeros} \\
        &\text{except possibly in the last }\min(\mu_i,\mu_j) \\ 
        &\text{columns of the last row of each }\mu_i\times\mu_j \text{ block} \\
        &\text{and } C_s \text{ is the block diagonal matrix of}\\
        &\text{companion matrices of } t^{\mu_{1,k}}(t-s)^{\mu_{2,k}}\}\,.
    \end{split}
\end{equation}

The uppertriangular subfamily $\TT_{\mu_1,\mu_2}^{+}$ with $s$-fibre
\begin{equation}
    \TT_{\mu_1,\mu_2}^{+,s} = \{B + C_s \in \TT_{\mu_1,\mu_2} : B \in\n\}
\end{equation}
where $\n\subset\mat(N)$ is the unipotent subalgebra of uppertriangular matrices.
\end{comment}

\acom{
    % or---as Joel pointed out, may be ok with:
The slice $\TT_\mu$ as defined in \cite{mirkovic2007quiver}, no change, \text{and} the family of slices $$\TT_{\mu_1,\mu_2}^{+,s} = \TT_\mu \cap \n + C_{\mu_1,\mu_2}^s$$ where $C_{\mu_1,\mu_2}^s$ is the block diagonal matrix of companion matrices of 
% 
\begin{equation}
    % C_s \text{ is the block diagonal matrix of companion matrices of } 
    t^{\mu_{1,k}}(t-s)^{\mu_{2,k}} \quad (1\le k\le m).
\end{equation}
% not sure about notation
Watch out for mixing up of $m$ and $n$.
}

The [family of] orbit[s $\OO_{\lambda_1,\lambda_2}$ fibred over $C = \AA$ with $s$-fibre]  
\begin{equation}
    \OO_{\lambda_1,\lambda_2}^s = \{A \in \cN_s : A \text{ is conjugate to } J_{\lambda_1}\oplus (sI_{N_2} + J_{\lambda_2})\}
\end{equation}
where $J_{\lambda_i}$ is the Jordan normal form of block type $\lambda_i$ and $I_{N_2}$ is the identity matrix in $\mat(N_2)$.
% 
\section{Exposition}

Anderson and Kogan conjectured in \cite{anderson2006algebra} \acom{or earlier?} that those MV polynomials which are cluster monomials for a Fomin--Zelevinsky cluster algebra structure on $\CC[N]$ are naturally expressible as determinants\dots
% and they conjectured a formula for many of them.

It's not clear how this work helps with/relates to AK/their conjectures.

The generalized MVy is interesting in its own right. \acom{is it?}

Computing fusion still hard but at least boiled down to linear algebra. Cf.\ fusion product as it appears in BD, FL, MV, AK, BFM.

Interesting combinatorics? Witnessing a product on SSYT.

Exchange relations only work on cluster modules where one is a mutation of the other (i.e.\ those corresponding to cluster monomials which are related by mutation). Of course this gives me everything. Up to $A_4$ as in type $A_5$ there exist indecomposable modules which are not cluster, so exchange relations do not apply. The hope (conjecture) is that this paper gives a way to compute on such modules. Cf.\ counterexample satisfying $\barD(c_Y) = \barD(b_Z) + 2\barD(b)$, where $b$ is (possibly) cluster, and suggesting that $c_Y = b_Z + 2b$. Can still consider $ext(M_Z)$. Say $b_Z^2 = b_{Z_1} + b_{Z_2}$. 

% No; $x*y=\sum z\Rightarrow y = \frac 1 x \sum z$. 

Representation theory? 

\section{Rising Action} % Problem

\begin{lemma} \label{le:Grl1l2}
    Let $ L \in \Grth^+ $.  Let $ s \in \AA, s \ne 0 $.  The following are equivalent:
    \begin{enumerate}
        \item $ L $ is in the image of the map $ \Gr^{\lambda_1, \lambda_2}_s \rightarrow \Grth^+$
        \item $ L \in G(\CC[t]) t^{\lambda_1} (t-s)^{\lambda_2}$
        \item The linear operator $ t $ on $ L_0/L$ has Jordan type $(0,\lambda_1), (s,\lambda_2)$.
    \end{enumerate}
\end{lemma}

\begin{lemma} \label{le:Wmu}
    Let $ L \in \Grth^+$.  The following are equivalent:
    \begin{enumerate}
        \item $ L \in \cW_\mu$.
        \item $ \{ [t^k e_i] : 0 \le k < \mu_i, 1 \le i \le m\}$ forms a basis for $ L_0/L$.
    \end{enumerate}
\end{lemma}

Since the above basis will occur often, we will denote it  $ \beta_\mu$.

\begin{lemma}
    Under the map $ \Grbd \rightarrow \Grth$, the image of $ S^{\mu_1, \mu_2}_\AA$ lands in $ \cW_\mu$.
\end{lemma}

\section{Climax}

\begin{theorem} 
The map $ T_\mu \rightarrow \Grth^+ \cap \cW_\mu $ given by $ A \mapsto g(A) $ is an isomorphism with inverse given by
$$ L \mapsto [t|_{L_0/L} ]_{\beta_\mu}$$
\end{theorem}

\jcom{Do we have a reference?}

For the next result, we will consider the ``intersection'' of $ \overline{\Gr}^{\lambda_1, \lambda_2}_\AA $ with $\cW_\mu$.  As $  \overline{\Gr}^{\lambda_1, \lambda_2}_\AA $ is not a subscheme of $ \Grth$ by this intersection, we really mean the preimage of $ \cW_\mu$ under the map $$ \overline{\Gr}^{\lambda_1, \lambda_2}_\AA  \hookrightarrow \Grbd \rightarrow \Grth$$
This is not a very serious abuse of notation, since the map $ \Grbd \rightarrow \Grth $ is almost injective.

In a similar way, we will write $ \overline{\OO}^{\lambda_1, \lambda_2}_\AA \cap T^\mu$, using the non-injective map $ \overline{\OO}^{\lambda_1, \lambda_2}_\AA \rightarrow M_N(\CC)$ (for example the fibre of this map over $ 0 $ is $ \CC $).

\begin{theorem} \label{th:OGrl}
    There is an isomorphism
    $$\overline{\OO}^{\lambda_1, \lambda_2}_\AA \cap T_\mu \cong \overline{Gr}^{\lambda_1, \lambda_2}_\AA \cap \cW_\mu $$
    given by $ (A,s) \mapsto (g(A), s)$.
\end{theorem}

\begin{theorem}
    The isomorphism from Theorem \ref{th:OGrl} restricts to an isomorphism
    $$ \overline{\OO}^{\lambda_1, \lambda_2}_\AA \cap T^{\mu_1, \mu_2, +}_\AA \cong \overline{Gr}^{\lambda_1, \lambda_2}_\AA \cap S^{\mu_1, \mu_2}_\AA$$
\end{theorem}


\section{Falling Action} % Resolution

\begin{theorem}
    Let $\lambda_i\ge\mu_i$ be dominant ($i=1,2$), $\mu = \mu_1 +\mu_2$, and $\lambda =\lambda_1+\lambda_2$. 
    % I have an irrational dislike of ending sentences with subscripts --- can we fix this?
    There is an isomorphism 
    \begin{equation}
        \overline{\OO}_{\lambda_1,\lambda_2}\cap\TT_{\mu_1,\mu_2} \to \overline\Gr_\AA^{\lambda_1,\lambda_2}\cap \cW_{\mu_1,\mu_2}
    \end{equation}
    got by taking a $\mu\times\mu$ block matrix $A$ in the $s$-fibre $\overline{\OO_{\lambda_1,\lambda_2}^s}\cap\TT_{\mu_1,\mu_2}^s$ on the left to the representative of the $s$-fibre on the right defined by  
    \begin{equation}
        \begin{split}
            g &= t^{\mu_1} (t-s)^{\mu_2} + a(t) \\
            a_{ij}(t) &= - \sum_{k=1}^{\mu_i} A^k_{ji} t^{k-1}
        \end{split}
    \end{equation}
    where $A^k_{ji}$ is the $k$th entry from the left of the last row of the $\mu_j\times\mu_i$ block of $A$. 
\end{theorem}

Let's call this the MVyBD isomorphism.

\begin{proof}
    The proof is fibre by fibre, so fix $s\ne 0$. \acom{Emphasize in the intro later (because this always confuses me) that by the $s$-fibre we really mean the $(0,s)$-fibre; i.e.\ its the BD Grassmannian over the second symmetric power of $C = \AA^1$; better just replace $s$-fibre by $(0,s)$-fibre everywhere it occurs.}
    \begin{enumerate}
        \item The map is well defined. In particular, it defines $\CC[t]$-lattices in $\CC(t)^m$. Moreover, these lattices break down to give pairs of lattices upon inverting $t$ or $t-s$ that have the right properties. [Copy Roger's proof]
        \item The inverse map is got by taking the matrix of multiplication by $t$.  More precisely, let $ L \in Gr^{BD} \cap \cW_\mu$.  We work with the quotient $\CC[t]^m/L$ just as in the ordinary MVy isomorphism---the only difference being $\CC\xt$ is replaced by $\CC[t]$.
\begin{enumerate}
    \item 
    We claim that 
    \begin{equation}
        \{[e_i],[te_i],\dots,[t^{\mu_{i}-1}e_i] : 1\le i \le m\}
    \end{equation}
    is a $\CC$-basis of $\CC[t]^m/L$.
    
    To see this, we use that $ L $ 
 has a $\CC[t]$-basis of the form 
    \begin{equation}
        v_i = t^{\mu_i} + \sum_{j>i} p_{ij}(t) e_j 
    \end{equation}
    with $\deg p_{ij}(t) < \mu_i = \mu_{1,i} + \mu_{2,i}$ ($1\le i\le m$).
    \acom{I don't know why this should be true. We might have to just define fibres of $\cW_{\mu_1,\mu_2}$ in this way?}
    \item $t\big|_{\CC[t]^m/L}$ will have two eigenvalues, 0 and $s$, and its generalized 0-eigenspace will have block type $\le \lambda_1$ while its generalized $s$-eigenspace will have block type $\le \lambda_2$. 
    To see this, note that there is a natural isomorphism
    $$\CC[[t]]^m/(L \otimes_{\CC[t]} \CC[[t]]) = \text{generalized $0$ eigenspace of $t$ on } \CC[t]^m/L$$
    carrying the action of $t $ to the action of $t$.
    
    The left hand side is the same thing as
    $$ \CC[[t]]^m / (L \otimes_{\CC[t]} \CC[[t]]) = (\CC[t]^m/L) \otimes_{\CC[t]} \CC[[t]] $$
    
    the defining fact that lattices satisfying Equation~\ref{eq:defGrBDlambda} equivalently satisfy 
    \begin{equation}
        \begin{split}
            t\big|_{\CC\xt^m/L_1}\text{ has Jordan type} \le \lambda_1 \\
            t\big|_{\CC\xt^m/L_2}\text{ has Jordan type} \le \lambda_2 
        \end{split}
    \end{equation}
    where recall 
    $L_i = L\otimes \CC\xt$ ??? 
    % $L_i = L\otimes \CC[(t-p_i)^{-1}]$ 
    and $p_1 = s$ while $p_2 = 0$. \acom{Somehow, restricting to an eigenspace is like inverting/forgetting the action of $t$ by any other generalized eigenvalue? Basic linear algebra? Joel?}
\end{enumerate}
    \end{enumerate}
\end{proof}



\begin{theorem}[Theorem 1 version 2]
    Let $\lambda_1,\lambda_2$ and $\mu$ 
    %be dominant
    be arbitrary, such that $\lambda = \lambda_1 + \lambda_2 \ge \mu$. Then there is an isomorphism 
    \begin{equation}
        \overline{\OO_{\lambda_1,\lambda_2}} \cap \TT_\mu \to \overline{\Grbd[2]^{\lambda_2,\lambda_2}}\cap\cW_\mu 
    \end{equation}
    defined by the same map as in Theorem 1.
\end{theorem}

\jcom{This is true as stated with the ``larger'' definition of $ \cW_\mu $.  In fact, for any $\lambda_1, \lambda_2$, it is true $ \overline{\Grbd[2]^{\lambda_2,\lambda_2}}\cap\cW_\mu $ is contained in a subset that we could call $ \cW_\mu^s$ which we could define as
$$
\cW_\mu^s = G_1[[t^{-1}]]t^\mu \cap G[t,t^{-1}, (t-s)^{-1}] / G[t]
$$
where we regard $G[t,t^{-1}, (t-s)^{-1}] / G[t] \subset G((t^{-1}))/G[t] $

The way to think about this is as follows: inside the thick affine Grassmannian we can consider the $G$-bundles trivialized away from just 0, $s$, or equivalently those lattices which become the standard lattice after tensoring with $ \CC[[t-a]] $ for any $ a \ne 0, s$.
}
\begin{corollary}
    The MVyBD isomorphism restricts to an isomorphism of subfamilies 
    \begin{equation}
        \overline{\OO_{\lambda_1,\lambda_2}}\cap\TT_{\mu_1,\mu_2}^+ \to \overline{(\Grbd[2])^{\lambda_1,\lambda_2}}\cap S_{\mu_1,\mu_2}\,. 
    \end{equation}
\end{corollary}

\begin{proof}
    Let $ A \in \overline{\OO_{\lambda_1,\lambda_2}}\cap\TT_{\mu_1,\mu_2}^+$ and let $ g $ be the polynomial matrix formed by the Mirkovic-Vybornov isomorphism.  Then the diagonal entries of $ g $ are $ t^{\mu_{1,k}} (t-s)^{\mu_{2,k}}$ and we can factor
    $$ g = (g t^{-\mu_1}(t-s)^{-\mu_2}) t^{\mu_1}(t-s)^{\mu_2} \in N[t, t^{-1}, (t-s)^{-1}] t^{\mu_1} (t-s)^{\mu_2}$$
    So we get containment in one direction.
    
    For the reverse containment, we choose $ [g] \in \overline{(\Grbd[2])^{\lambda_1,\lambda_2}}\cap S_{\mu_1,\mu_2}$.  By the lemma below, $[g] \in \cW_\mu$ and thus it lies in the image of our map and we are done.
\end{proof}

% Define $S_{\mu_1,\mu_2}^s = N_-\xT[t^{-1}] t^{\mu_1} (t-s)^{\mu_2}$. 

\acom{is it a fibre of $S_{\mu_1,\mu_2}$ defined above?}

We could also make the following claim. 


\begin{lemma}[KWWY14]
    Let $\mu$ be dominant. Then 
    \begin{equation}
        N_{-}\xT[t^{-1}] L_\mu = N_1\xt[t^{-1}]L_\mu
    \end{equation}
    \acom{where I am not sure about the double brackets.}
\end{lemma}

\begin{lemma}%[Roger's lemma]
    Let $\mu_1,\mu_2$ be dominant and let $s\in \AA^1 - \{0\}$. Then 
    \begin{equation}
        S_{\mu_1,\mu_2}^s \subset \cW_\mu 
    \end{equation}
    where $\mu = \mu_1 + \mu_2$.
\end{lemma}

\begin{proof}
    % Copy Roger's proof.
    We have
\[
\begin{split}
    S_{\mu_1, \mu_2} & = N((t^{-1}))t^{\mu_1}(t-s)^{\mu_2} \\
     & \subset T_1[[t^{-1}]] N((t^{-1})) t^{\mu_1} (t-s)^{\mu_2} \\
     & = T_1[[t^{-1}]] N_1[[t^{-1}]] t^{\mu_1} (t-s)^{\mu_2} \qquad \text{\cite[Lemma 2.3]{kamnitzer2014yangians}}\\
     & = B_1[[t^{-1}]] t^{\mu_1} (t-s)^{\mu_2} \\
     & = B_1[[t^{-1}]] t^{\mu_1 + \mu_2} \\
     & \subset G_1[[t^{-1}]] t^{\mu_1 + \mu_2} \\
     & = W_{\mu_1 + \mu_2}
\end{split}
\]
where $B_1[[t^{-1}]] t^{\mu_1} (t-s)^{\mu_2} = B_1[[t^{-1}]] t^{\mu_1 + \mu_2}$ since 
\[
\frac{t}{t-s} = 
1 + \frac{s}{t} + \frac{s^2}{t^2} + \cdots 
\in B_1[[t^{-1}]].
\]
\end{proof}

\section{Denouement}

As an application we can compute fusion of stable MV cycles of type $\alpha_i$ for any $i \in I$. What about more general weights? Having $\kpf > 1$.

\begin{proposition}
    % Given two MV cycles $Z_\tau$ and $Z_\sigma$ of type\dots 
    Let $Z_{i} = Z_{\tau_i}$  be an MV cycle of type $\lambda_i\in P_+$ and weight $\mu_i\in P$ ($i = 1,2$). Then 
    \begin{equation}
        % Z_\tau\ast Z_\sigma 
        m_{\lambda_1\lambda_2} ([Z_1]\otimes[Z_2]) = \sum_{Z\in\cZ(\lambda)_\mu} i\left(
            Z,\pi^{-1}(0)\cdot\overline{Z_1\times Z_2\times U}
        \right) [Z]
    \end{equation}
    is found by\dots 
    In particular, the multiplicity of $Z_\tau$ in the product of the zero fiber 
    % (divisor class) 
    and the 
    % (Zariski) 
    closure of the family $Z_1 \times Z_2 \times U$ is equal to the multiplicity of $X_\tau$ among 
    \begin{equation}\label{eq:tabmult}
        \irr \lim_{s\to 0} \overline\OO^s_{\lambda_1,\lambda_2} \cap \TT_{\mu_1,\mu_2}^{+,s}
    \end{equation}
    and since stable MV cycles can be represented by ordinary ones the multiplicities in 
    $$
    b_{Z_1}b_{Z_2} = \sum_{Z\in\cZ(\infty)_{-\nu_1 - \nu_2}} i \left(
        Z, \pi^{-1}(0) \cdot \overline{Z_1 \times Z_2 \times U}
    \right) b_Z 
    $$
    can also be deduced from \cref{eq:tabmult} for appropriate choices of $\lambda_i$ and $\mu_i$ ($i = 1,2$).
\end{proposition}
\acom{or Corollary?}

\begin{conjecture}
    % Let $Z_i \subset \overline{S^{\nu_i}\cap S^0_-}$ be an MV cycle of weight $\nu_i$ ($i = 1,2$) and put $\nu = \nu_1 + \nu_2$. 
    Let $\tau$ be the tableau of shape $\lambda$ and weight $\mu$ whose Lusztig datum is equal to the sum of the Lusztig data of $\tau_1$ and $\tau_2$. 
    % Then $i(\tau, \pi^{-1}(0) \cdot \overline{\tau_1 \times \tau_2 \times U})$ is equal to 1. 
    Then 
    \begin{equation}
        i(\tau, \pi^{-1}(0) \cdot \overline{\tau_1 \times \tau_2 \times U}) = 1 \,. 
    \end{equation}
\end{conjecture}

\bibliographystyle{alpha}
\bibliography{mvybd}

\end{document}