\documentclass{article}
\usepackage{basic}

\title{Examples Compendium}
\author{Roger Bai, Anne Dranowski}
% \date{October 2020}
% , Joel Kamnitzer
\date{Last edit: approx.\ \today}

\begin{document}

\maketitle
\tableofcontents

% \section{Examples}
\section{\today}
Let $Z$ be the MV cycle of weight $\alpha_3$ with Lusztig datum $(000,00,1)$ and $Z'$ the MV cycle of weight $\alpha_{1,2}$ with Lusztig datum $(010,00,0)$. 

In terms of tableax the fusion $Z\ast Z'$ can be encoded as 
\begin{align}
    \young(1,2,4) \ast \young(3) &= \young(13,2,4) + \young(14,2,3) \\
    S_3 \ast (1\leftarrow 2) &= (1\leftarrow 2 \to 3) + P_3 \\ 
    (000,00,1)\ast (010,00,0) &= (010,00,1) + (001,00,0) \\
    &= (A_0 , A_5, A_1A_4 - A_2A_3) \sqcup (A_0 , A_1, A_3) 
\end{align}
where the ideals in line 3 are given in coordinates on matrices of the form
\[
    \left[\begin{BMAT}(e){c;c;c;c}{c;c;c;c} 
        0 & A_0 & A_1 & A_2\\
        0 & 0 & A_3 & A_4\\
        0 & 0 & s & A_5\\
        0 & 0 & 0 & 0
        \end{BMAT}\right]
\]

Next let $Z$ be the MV cycle of weight $\alpha_{1,3}$ having Lusztig datum $(100,01,0)$ and $Z'$ the MV cycle of weight $\alpha_2$ having Lusztig datum $(000,10,0)$. 

Again, in terms of tableaux, $Z\ast Z'$ is given by 
\begin{align}
    \young(2,4) \ast \young(1,3) &= \young(13,24) + \young(12,34) \\
    (1\to 2\leftarrow 3) \ast S_2 &= P_2 + (2\leftarrow 3) \oplus (1 \to 2) \\
    (100,01,0) \ast (000,10,0) &= (010,01,0) + (100,11,0) 
\end{align} 

TODO: fill in the ideals. 

Now take $Z$ to be the MV cycle of weight $\alpha_{1,3}$ with Lusztig datum $(010,00,1)$ and leave $Z'$ as above. Then 
\begin{align}
    \young(13,2,4) \ast \young(11,23) &= \young(1113,223,4) + \young(1113,224,3) \\
    (1 \leftarrow 2 \to 3) \ast S_2 &= (2\to 3) \oplus (1\leftarrow 2) + P_2 \\
    (010,00,1) \ast (000,10,0) &= (010,10,1) + (010,01,0)
\end{align}
TODO: fill in the ideals. 

Also computed: 
\begin{align}
    \young(1,2,4) \ast \young(12,3) &= \young(112,24,3) + \young(112,23,4) \\
    (000,00,1) \ast (100,10,0) &= (100,01,0) + (100,10,1) \\
    S_3 \ast (1\to 2) &= (1 \to 2 \leftarrow 3) + (1 \to 2 \to 3) \\
    &= (A_0,A_5,A_4,A_2) \sqcup (A_0,A_5,A_4,A_6) 
\end{align}
in 
\[
    \left[\begin{BMAT}(e){cc;cc;c;c}{cc;cc;c;c} 
        0 & 1 & 0 & 0 & 0 & 0\\
        0 & s & A_0 & A_1 & A_2 & A_3\\
        0 & 0 & 0 & 1 & 0 & 0\\
        0 & 0 & 0 & s & A_4 & A_5\\
        0 & 0 & 0 & 0 & s & A_6\\
        0 & 0 & 0 & 0 & 0 & 0
        \end{BMAT}\right]    
\]
TODO: add explanation, double check the ideals. 

Finally, we guess that the fusion $2\alpha_1 \ast 2\alpha_2$ is encoded by the tableau equation 
\[
\young(22) \ast \young(11,33) = \young(1122,33) + \young(1123,23) + \young(1133,22)     
\]
Remains to check this and make sense of the middle term. 

% 1
% 2 0 
% 2 1 0 
% 2 1 0 0 

% 2
% 3 1 
% 3 1 1 
% 3 2 1 0 

% 2 
% 3 1 
% 3 2 0 
% 3 2 1 0 

\section{Disjoint, non-dominant weight}

\begin{example}
$\lambda_1 = (1,0,0)$, $\lambda_2 = (1,1,0)$, $\mu_1 = (0,1,0)$, $\mu_2 = (1,0,1)$. \jcom{$\mu = \mu_1 + \mu_2$ determines the blocks we have on the RHS of the BD MVy isomorphism.} 
\[
\left[\begin{BMAT}(e){c;c;c}{c;c;c} 
    s & A_0 & A_1\\
    0 & 0 & A_2\\
    0 & 0 & s
\end{BMAT}\right]    
\]
% 
% In the non-BD case, MVy establish
% $$
% \overline{\Gr^\lambda}\cap \cW^\mu \to 
% \left\{ X = 
% \left[\begin{BMAT}{cc|cc|c}{cc|cc|c}
% 0 & 1 & 0 & 0 & 0 \\
% * & * & * & * & * \\
% 0 & 0 & 0 & 1 & 0 \\
% * & * & * & * & * \\
% * & 0 & * & 0 & * 
% \end{BMAT}\right] \big| X \in \overline{\OO}_\lambda
% \right\}
% $$

% In the BD case the RHS will consist of the same block like matrices $X$ but now having eigenvalues $s, 0$ such that $X - s\big|_{E_s}\in \OO_{\lambda_2}$ 
\end{example}

\section{Some multiplicity}

\begin{example}[Joel's exercise] 
    It would be good to do an example where we will see some multiplicity in the fusion product. I think that the simplest example would be the fusion product of the MV cycles for $\SL_3$ of weights $2\alpha_1$ and $2\alpha_2$.  
    Following the notation from the mvbasis paper, this would correspond to the following multiplication: $$x^2 y^2 = (xy-z)^2 + 2(xy - z) z + z^2\,.$$

For this example, I think we need $\lambda_1 = (2,0,0)$, $\lambda_2 = (2,2,0)$, $\mu_1 = (0,2,0)$, $\mu_2 = (2,0,2)$. 
\[
\yng(2) = \{\young(22)\} \qquad \yng(2,2) = \{\young(11,33)\} 
\]
Then 
\[
A = \left[\begin{BMAT}(e){cc;cc;cc}{cc;cc;cc} 
    0 & 1 & 0 & 0 & 0 & 0\\
    -s^2 & 2s & A_0 & A_1 & A_2 & A_3\\
    0 & 0 & 0 & 1 & 0 & 0\\
    0 & 0 & 0 & 0 & A_4 & A_5\\
    0 & 0 & 0 & 0 & 0 & 1\\
    0 & 0 & 0 & 0 & -s^2 & 2s
    \end{BMAT}\right]
\]
needs $A-s\big|_{\CC^2\cap E_s}$ has Jordan type $(2)$, $A\big|_{\CC^4\cap E_0}$ has Jordan type $(2)$, and $A-s\big|_{E_s}$ has Jordan type $(2,2)$. 
So 
\[
E_0 = \{A_0 \e_1 + s^2\e_3\leftarrow A_0 \e_2 + s^2 \e_4 + (A_0 + A_1 + \frac {2A_0}  s  ) \e_1 + s^2 \e_3 \}   
\]
while 
\begin{gather}
E_s^1 = \{\e_1 + s \e_2\leftarrow \e_1 + (1 + s) \e_2 \}  \\
E_s^2 = \{ -\frac{A_2 + sA_3}{A_0 + sA_1}\e_3 - \frac{sA_2 + s^2 A_3}{A_0 + sA_1} \e_4 + \e_5 + s \e_6  \leftarrow\}
\end{gather}
if 
\begin{gather}
    0 = A_4 + sA_5 \\
    \to A_0 A_4 \text{ as } s \to 0 
\end{gather}
and 


% Adding 2020-12-30 14:03:05: 
Let's also try  
\[
\begin{aligned}
    \lambda_2 &= (4,0,0) & \lambda_1 &= (4,4,0) & \lambda &= (8,4,0)\\
    \mu_2 &= (2,2,0)     & \mu_1 &= (4,2,2)     & \mu &= (6,4,2)
\end{aligned}    
\]
Then 
\[
\left[\begin{BMAT}(e){cccccc;cccc;cc}{cccccc;cccc;cc} 
    0 & 1 & 0 & 0 & 0 & 0 & 0 & 0 & 0 & 0 & 0 & 0\\
    0 & 0 & 1 & 0 & 0 & 0 & 0 & 0 & 0 & 0 & 0 & 0\\
    0 & 0 & 0 & 1 & 0 & 0 & 0 & 0 & 0 & 0 & 0 & 0\\
    0 & 0 & 0 & 0 & 1 & 0 & 0 & 0 & 0 & 0 & 0 & 0\\
    0 & 0 & 0 & 0 & 0 & 1 & 0 & 0 & 0 & 0 & 0 & 0\\
    0 & 0 & 0 & 0 & -s^2 & 2s & A_0 & A_1 & A_2 & A_3 & A_4 & A_5\\
    0 & 0 & 0 & 0 & 0 & 0 & 0 & 1 & 0 & 0 & 0 & 0\\
    0 & 0 & 0 & 0 & 0 & 0 & 0 & 0 & 1 & 0 & 0 & 0\\
    0 & 0 & 0 & 0 & 0 & 0 & 0 & 0 & 0 & 1 & 0 & 0\\
    0 & 0 & 0 & 0 & 0 & 0 & 0 & 0 & -s^2 & 2s & A_6 & A_7\\
    0 & 0 & 0 & 0 & 0 & 0 & 0 & 0 & 0 & 0 & 0 & 1\\
    0 & 0 & 0 & 0 & 0 & 0 & 0 & 0 & 0 & 0 & 0 & 0
    \end{BMAT}\right]    
\]
Note that there is only one SSYT of shape $\lambda_i$ and weight $\mu_i$
\[
\tau_2 = \young(1122) \qquad \tau_1 = \young(1111,2233)    
\]
Note also that 
$$\begin{aligned}
    t^4(t-s)^2 &= t^4 (t^2 - 2st + s^2) = t^6 - 2st^5 + s^2 t^4 \\
    t^2 (t-s)^2 &= t^4 - 2st^3 + s^2 t^2  % \\
    % t^2 &= t^2 
\end{aligned}$$
Elements of $\TT_{\mu_1,\mu_2}^+$ will take the form 
\[
A = \left[\begin{BMAT}(e){cccccc:cccc:cc}{cccccc:cccc:cc}
    & 1 & & & & & & & & & & \\
    & & 1 & & & & & & & & & \\
    & & & 1 & & & & & & & & \\
    & & & & 1 & & & & & & & \\
    & & & & & 1 & & & & & & \\
    & & & & -s^2 & 2s & 0 & 0 & a_3 & a_4 & b_1 & b_2 \\
    & & & & & & & 1 & & & & \\
    & & & & & & & & 1 & & & \\
    & & & & & & & & & 1 & & \\
    & & & & & & & & -s^2 & 2s & c_1 & c_2 \\
    & & & & & & & & & & & 1 \\
    & & & & & & & & & & & 
\end{BMAT}\right]
\]
The tableau tells us for each $1\le i\le 12$
\[
    \text{Jordan type of } A\big|_{\Sp(e_1,\dots,e_i)\cap E_0} \text{ is shape of }\tau_1\big|_{\text{first $i$ boxes}}
\]
So take $i = 10$. Then $A$ restricted to $\CC^{10} \cap E_0$ should have Jordan type $(4,2)$. 
\acom{How to do it box by box? $s$ columns somehow correspond to $\tau_2$ boxes.}
Therefore $a_1 = 0$. The 4-cycle is obvious. For the 2-cycle we require $a_2 = 0$. 
\[
    \begin{aligned}
e_1 &\leftarrow e_2 \leftarrow e_3 \leftarrow e_4 \\
e_7 &\leftarrow e_8 \\
% e_{11} &\leftarrow e_{12} 
    \end{aligned}    
\]
Now looking at all of $A$ we have continue our 2-cycle to a 4-cycle. Roger found 
\[
    e_{10} - \left(\frac{2s}{c_1}  + \frac{s^2 c_2}{c_1^2}\right) e_{11} + \frac{s^2}{c_1} e_{12}
\]
This requires $c_1^2 a_4 + c_1b_2s^2 - 2sc_1b_1-s^2 c_2 = 0$ and $a_3 c_1 + s^2 b_1 = 0$.

Now looking for the $s$-eigenspace, we expect $A-s\big|_{\CC^6\cap E_s}$ to have Jordan type $2$. The kernel is spanned by 
$e_1 + se_2 + s^2 e_3 + s^3 e_4 + s^4 e_5 + s^5 e_6$. It is continued to a 2-cycle by/the 2-cycle is generated by
$-\frac 5 s e_1 - 4 e_2 - 3 s e_3 - 2 s^2 e_4 - s^3 e_5$. 
The 3-cycle is maybe 
$(1/s^2, -4/s, -8, 5s, 3s^2, 2s^3, -s^2/a_3, -s^3/a_3, -s^4/a_3, -s^5/a_3)$ padded with zeros. 
% The vectors $e_5,e_6, e_9, e_{10}$ are left to form the $s$-eigenspace. 
% Let's figure this out next. 


\[
A-s = \left[\begin{BMAT}(e){cccccc:cccc:cc}{cccccc:cccc:cc}
    -s & 1 & & & & & & & & & & \\
    & -s & 1 & & & & & & & & & \\
    & & -s & 1 & & & & & & & & \\
    & & & -s & 1 & & & & & & & \\
    & & & & -s & 1 & & & & & & \\
    & & & & -s^2 & s & 0 & 0 & a_3 & a_4 & b_1 & b_2 \\
    & & & & & & -s & 1 & & & & \\
    & & & & & & & -s & 1 & & & \\
    & & & & & & & & -s & 1 & & \\
    & & & & & & & & -s^2 & s & c_1 & c_2 \\
    & & & & & & & & & & -s & 1 \\
    & & & & & & & & & & & -s 
\end{BMAT}\right]
\]
\end{example}

\begin{comment}
\section{Getting a feel for things}
\begin{example}
    Let 
    \[
    \begin{aligned}
        \mu_1 &= (2) & \lambda_1 &= & \mu &= (5) \\
        \mu_2 &= (3) & \lambda_2 &= & \lambda &= 
    \end{aligned}
    \]
    Consider the companion matrix $C(p)$ of 
    $$p(t) = (t-s)^3t^2 = (t^3 - 3t^2 s + 3t s^2 - s^3)t^2 = t^5 - 3 t^4 s + 3t^3 s^2 - t^2 s^3$$ 
    Let $X = C(p)^T$ so 
    \[
    X = \left[\begin{BMAT}(e){ccccc}{ccccc}
        0 & 1 & & & \\ 
        & 0 & 1 & & \\
        &   & 0 & 1 & \\ 
        &   &   & 0 & 1 \\
       0 & 0 & s^3 & -3s^2  & 3s  
    \end{BMAT}\right]    
    \]
    Ask that $X\big|_{E_0}$ has Jordan type $\lambda_1$ and $X-s\big|_{E_s}$ has Jordan type $\lambda_2$. In this rank 1 case we are forced to take $\lambda_i = \mu_i$. 
    
    So what are the generalized eigenspaces $E_i$ ($i = 1,2$)? Note $\dim E_0 = 2$ and $\dim E_s = 3$. 
    
    \acom{The basis
    \[
     [1], [t], [\bf 1], [\bf t], [\bf t^2]    
    \] 
    with $t[t] = 0$ and $t[{\bf t^2}] =   s^3 [{\bf 1}] - 3s^2 [{\bf t}] + 3s[{\bf t^2}]$ \textit{is not the correct basis to consider}.
    Hence my confusion of yore: what we would like is $t[t] = 0$ no? what the matrix is telling us is that $t[t] = [{\bf 1}]$. Can we still speak of \textit{two} generalized eigenspaces?}  % [3st^2 - 3s^2 t + s^3]

    Rather, take $B$ to be the basis $b_1 = e_1$, $b_2 = e_2$, $b_5 = e_5$, $b_4 = Xe_5$, $b_3 = X^2e_5)$. In this basis
    \[
    X_B = [X(b_i)] = \begin{bmatrix}
        0 & 1 & 0 & 0 & 0 \\
        0 & 0 & 0 & 0 & 0 \\
        0 & 0 & s & 1 & 0 \\
        0 & 0 & 0 & s & 1 \\
        0 & 0 & 0 & 0 & s
    \end{bmatrix}
    \]

    % I think this demonstrates why Roger's idea is probably the right thing to do. Namely, consider instead $T_{\mu_1,\mu_2}$ whose elements are 
    % \[
    %     X = \left[\begin{BMAT}(e){cc|ccc}{cc|ccc}
    %         0 & 1 & & & \\ 
    %         & 0 & x & y & z \\
    %         &   & 0 & 1 & \\ 
    %         &   &   & 0 & 1 \\
    %        0 & 0 & s^3 & -3s^2  & 3s  
    %     \end{BMAT}\right]    
    % \] 
    % such that $T_{\mu_1,\mu_2}\big|_{s = 0} = T_\mu$ whatever that means... 
\end{example}
\end{comment}

\section{Simple root weights, things working}
\begin{example}
    Let 
    \[
    \begin{aligned}
        \mu_1 &= (3,1,1) & \lambda_1 &= (3,2,0) & \mu = (3,3,1) \\
        \mu_2 &= (0,2,0) & \lambda_2 &= (2,0,0) & \lambda = (5,2,0)
    \end{aligned}    
    \]
    and consider the companion matrices of 
    $$
    \begin{aligned}
        p_1(t) &= t^3 & p_2(t) &= t(t-s)^2 = t^3 - 2st^2 + s^2 t & p_3(t) &= t
    \end{aligned} % = t(t^2 - 2st + s^2) 
    $$
    \[
    X = \left[
        \begin{BMAT}(e){ccc;ccc;c}{ccc;ccc;c}
            0 & 1 & 0 & 0 & 0 & 0 & 0 \\
            0 & 0 & 1 & 0 & 0 & 0 & 0 \\
            % a & b & c & d & e & f & g \\ 
            0 & 0 & 0 & d & e & f & g \\ 
            0 & 0 & 0 & 0 & 1 & 0 & 0 \\
            0 & 0 & 0 & 0 & 0 & 1 & 0 \\
            0 & 0 & 0 & 0 & -s^2 & 2s & k \\
            0 & 0 & 0 & 0 & 0 & 0 & 0
        \end{BMAT}
        \right]    
    \]
    % \[
    %     \begin{BMAT}(e){ccc;ccc;c}{ccc;ccc;c} 
    %         0 & 1 & 0 & 0 & 0 & 0 & 0\\
    %         0 & 0 & 1 & 0 & 0 & 0 & 0\\
    %         0 & 0 & 0 & A_0 & A_1 & A_2 & A_3\\
    %         0 & 0 & 0 & 0 & 1 & 0 & 0\\
    %         0 & 0 & 0 & 0 & 0 & 1 & 0\\
    %         0 & 0 & 0 & 0 & -s^2 & 2*s & A_4\\
    %         0 & 0 & 0 & 0 & 0 & 0 & 0
    %     \end{BMAT}    
    % \]
\end{example}

\begin{example}
    Let
    \[
    \begin{aligned}
        \lambda_1 &= (3,2,0) & \mu_1 &= (3,1,1) \\
        \lambda_2 &= (2,0,0) & \mu_2 &= (1,1,0)
    \end{aligned}
    \]
    so the first MV cycle $Z_1 \cong \PP^1$ has MV polytope $\Conv\{0, \alpha_1\}$ and the second MV cycle $Z_2 \cong \PP^1$ has MV polytope $\Conv\{0, \alpha_2\}$. Their fusion product corresponds to two $\PP^2$'s intersecting along a $\PP^1$. That is, we have the fusion product
    \[
    Z_1 * Z_2 = Z_+ + Z_-
    \]
    where $Z_+ \cong Z_- \cong \PP^2$.
    We have 
    \[
    X = \begin{bmatrix}
    0 & 1 \\
      & 0 & 1 \\
      &   & 0 & 1 \\
      &   &   & s & a & b & c \\
      &   &   &   & 0 & 1 & 0 \\
      &   &   &   & 0 & s & d \\
      &   &   &   &   &   & 0
    \end{bmatrix}
    \]
    The $0$-generalized eigenspace $E_0$ of $X$ is $5$-dimensional, containing a 3-cycle and a 2-cycle.
    The 3-cycle is 
    \[
    \{X^2 e_3 , X e_3, e_3\} = \{e_1, e_2, e_3\}.
    \]
    To obtain another vector in $\ker X$, either $a = 0$ or $c = d = 0$, but the latter case cannot give a 2-cycle as $e_7 \notin \im X$. Then $a = 0$ and we obtain a 2-cycle
    \[
    \left\{X \left( e_6 -\frac{s}{d}e_7 \right), e_6 - \frac{s}{d}e_7 \right\}
    = \left\{e_5, e_6 - \frac{s}{d}e_7 \right\}.
    \]
    We also obtain the equations $b \neq 0$, $d \neq 0$, and $sc - bd = 0$ from this.
    
    For the $s$-generalized eigenspace $E_s$, we need $a + sb \neq 0$ to obtain a 2-cycle, which can be taken as 
    \[
    \begin{split}
    & \left\{ 
    (X -sI) \left( e_2 + 2s e_3 + 3s^2 e_4 + \frac{s^3}{a+sb} e_5 + \frac{s^4}{a+sb} e_6 \right), \right. \\ 
    & \quad \qquad \qquad  \left. e_2 + 2s e_3 + 3s^2 e_4 + \frac{s^3}{a+sb} e_5 + \frac{s^4}{a+sb} e_6 
    \right\} \\
    = & \left\{ 
    e_1 + s e_2 + s^2 e_3 + s^3 e_4, 
    e_2 + 2s e_3 + 3s^2 e_4 + \frac{s^3}{a+sb} e_5 + \frac{s^4}{a+sb} e_6
    \right\}
    \end{split}
    \]
    The minimal polymonial is $X^3 (X-sI)^2$, which when equated to 0 gives again the equation $cs-bd = 0$.
    Thus the defining equations are 
    \[
    \{ a = 0, cs-bd = 0\}.
    \]
    When we take $s = 0$, we get the equations
    \[
    \{a = 0, bd = 0\}
    \]
    which corresponds to two $\AA^2$'s intersecting along an $\AA^1$. This is indeed an open subset of $\PP^2 \cup_{\PP_1} \PP^2$, as required.
\end{example}

\begin{example}[Continued\dots]
    The matrix $X$ from the previous example defines (under MVy) the matrix 
    \[
        g = \begin{bmatrix}
            (t-s)t^3 \\
            -bt & (t-s)t \\
            -c & -d & t 
        \end{bmatrix}
    \]
    in $G(\cO)$. 
    Indeed the various blocks of $X$ are in a precise sense the companion matrices of the polynomial entries of $g$
    % First let's double check the inverse map. 

    In $\Gr$ the element $g$ defines the lattice 
    \[
        gL_0 = \CC[t]\langle
        (t-s)t^3e_1 - (a+bt) e_2 - ce_3 , (t-s)te_2 - de_3, te_3 
        \rangle
    \]
    and computing the matrix of the action of $t$ on the quotient $L_0/L$ in the basis 
    \[
    \{
        [e_1],[te_1],[t^2e_1],[t^3e_1],[e_2],[te_2],[e_3]
    \}    
    \]
    recovers $X$ up to a transpose of course. 

    Now let's see what we get when we invert $t$ and $t-s$ respectively. 

    First let's invert $t$ by considering $L_2 = L\otimes\CC\xt[t-s]$. 
    \[
        L_2  = \CC[t,t^{-1}]\langle(t-s) e_1 -\frac{a+bt}{t^3} e_2 - \frac c {t^3} e_3, (t-s) e_2 - \frac d t e_3, e_3 \rangle
    \]
    so in $L_0/L_2$ we have 
    \[
    t[e_1] = s[e_1] + \frac{a+bt}{t^3} [e_2] + \frac c {t^3}  [e_3] \qquad t[e_2] = s[e_2] + \frac d t[e_3]    \qquad [e_3] = 0 
    \]
    and
    \[
    \left[t\big|_{L_0/L_2}\right]_{\{[e_1],[e_2]\}} = \begin{bmatrix}
        s \\
        \frac{a+bt}{t^3} & s 
    \end{bmatrix}
    \]
    which upon subtracting $s I$ gives a matrix having block type $\mu_2$ and Jordan type $\lambda_2 = (2)$ assuming $\frac{a+bt}{t^3}\ne 0$. 
    
    Next let's invert $t-s$ by considering $L_1 = L\otimes\CC\xt$. 
    \[
    \begin{aligned}
        L_1 &= \CC[t,(t-s)^{-1}]\langle
        t^3 e_1 - \frac{a+bt}{t-s}e_2 - \frac c {t-s} e_3, te_2 - \frac d {t-s} e_3 , te_3 
        \rangle \\
        &= \langle
        t^3 e_1 - \frac{a}{t-s}e_2 - \frac b{t-s} te_2 - \frac c {t-s} e_3, te_2 - \frac d {t-s} e_3 , te_3 
        \rangle
    \end{aligned}
    \]
    so in $L_0/L_1$ we have 
    \begin{align*}
        t[e_1] &= [te_1] \\
        t[te_1] &= [t^2 e_1] \\ 
        t[t^2e_1] &= \frac{a}{t-s}[e_2] + \frac{b}{t-s} t[e_2] + \frac c {t-s} [e_3] \\
                  &= \frac b{t-s} \frac d{t-s} [e_3] + \frac c {t-s} [e_3] \\ 
                  &= \frac {bd + (t-s) c} {(t-s)^2} [e_3] \\
                  &= \frac{bd - sc}{(t-s)^2}[e_3] + \frac c {(t-s)^2} t[e_3] = 0 \\ 
        t[e_2] &= \frac d {t-s} [e_3] \\ 
        t[e_3] &= 0 
    \end{align*}
    and 
    \[
    \left[t\big|_{L_0/L_1}\right]_{\{[e_1],[te_1],[t^2e_1],[e_2],[e_3]\}} = \begin{bmatrix}
        0 \\
        1 & 0 \\
        0 & 1 & 0 \\
          & & 0 & 0 \\
          & & 0 & \frac d {t-s} & 0  
    \end{bmatrix}
    \]
    taking transpose 
    \[
        \begin{bmatrix}
            0 & 1\\
              & 0 & 1\\
              &   & 0 &  &  \\
              & & & 0 & \frac d {t-s} \\
              & & & & 0  
        \end{bmatrix}
    \]
    which has block type $\mu_1$ and Jordan type $\lambda_1 = (3,2)$ assuming $d\ne 0$.

    I have used the relations Roger found (and I checked) $a = 0$ and $cs - bd = 0$ in the calculations above. 
    
    To sum up, the pair of matrices above should contain the same information as the matrix from the previous example 
    \[
    \begin{bmatrix}
    0 & 1 \\
      & 0 & 1 \\
      &   & 0 & 1 \\
      &   &   & s &  & b & c \\
      &   &   &   & 0 & 1 & 0 \\
      &   &   &   & 0 & s & d \\
      &   &   &   &   &   & 0
    \end{bmatrix} \Leftrightarrow \left( \begin{bmatrix}
            0 & 1\\
              & 0 & 1\\
              &   & 0 &  &  \\
              & & & 0 & \frac d {t-s} \\
              & & & & 0  
        \end{bmatrix} , \begin{bmatrix}
        s & \frac b {t^2} \\ & s 
        \end{bmatrix}\right) 
    \] 
\end{example}

{\bf Wish.} Given $L$ in $\Grbd$ define a map to $T_\mu$ just like MVy by taking $[t\big|_{L_0/L}]$ and use the fact that $[t\big|_{L_0/L_i}]$ for $i = 1,2$ are companion matrices of the right type, piece together two MVy isomorphisms to make a BD MVy iso. 

Equivalent linear algebra question(?): If $p(t,t-s) = p_1(t)p_2(t-s)$ then how are $C(p_1)$, $C(p_2)$, and $C(p)$ related? 
% I think the answer is basically this theorem \url{https://en.wikipedia.org/wiki/Structure_theorem_for_finitely_generated_modules_over_a_principal_ideal_domain} 
\newpage
\section{Non simple root weights}
\begin{example}[Anne]
    Let $G = \SL_3$ and $\uvi = 121$.
    
    Take $n_\bullet^1 = (1,0,0)$, and $n_\bullet^2 = (1,0,1)$ or $(0,1,0)$. So 
    \[
        \begin{aligned}
            \mu_1 &= (2,2,1) & \mu_2 &= (1,1,1) & \mu &= (3,3,2)\\
            \lambda_1 &= (3,1,1) & \lambda_2 &= (2,1,0) & \lambda &= (5,2,1)
        \end{aligned}
    \]
    Note
    \[
    \tau(1,0,0) = \young(112,2,3) \qquad \tau(1,0,1) = \young(12,3) \qquad \tau(0,1,0) = \young(13,2)    
    \]
    
    \acom{We should show that order does not matter; i.e.\ swapping indices on $\lambda$'s and $\mu$'s produces the same result.}

    $\TT_{\mu_1,\mu_2}^+\cap\OO_{\lambda_1,\lambda_2}$ is made up of elements of the form
    \[
        A = \left[\begin{BMAT}(e){ccc;ccc;cc}{ccc;ccc;cc} 
            0 & 1 & 0 & 0 & 0 & 0 & 0 & 0\\
            0 & 0 & 1 & 0 & 0 & 0 & 0 & 0\\
            0 & 0 & s & A_0 & A_1 & A_2 & A_3 & A_4\\
            0 & 0 & 0 & 0 & 1 & 0 & 0 & 0\\
            0 & 0 & 0 & 0 & 0 & 1 & 0 & 0\\
            0 & 0 & 0 & 0 & 0 & s & A_5 & A_6\\
            0 & 0 & 0 & 0 & 0 & 0 & 0 & 1\\
            0 & 0 & 0 & 0 & 0 & 0 & 0 & s
            \end{BMAT}\right]
    % \begin{bmatrix}
    %     0 & 1 & 0 \\
    %     0 & 0 & 1 \\
    %     0 & 0 & s & 0 & a_2 & a_3 & b_1 & b_2 \\
    %       &   &   & 0 & 1 & 0 \\
    %       &   &   & 0 & 0 & 1 \\
    %       &   &   & 0 & 0 & s & c_1 & 0 \\
    %       &   &   &   &   &   & 0 & 1 \\
    %       &   &   &   &   &   & 0 & s 
    % \end{bmatrix}    
    \]
As usual, denote by $E_e$ the generalized $e$-eigenspace of $A$. $A\big|_{\CC^3\cap E_0}$ should have Jordan type $(2)$. The obvious 2-cycle is generated by $e_2$: $\{e_2, A e_2\}$. $A\big|_{\CC^3\cap E_s}$ should have Jordan type $(1)$. We take $e_1 + se_2 + s^2 e_3 \in\Ker(A-s)$. Next $A\big|_{\CC^6\cap E_0}$ should have Jordan type $(3,1)$ while $A\big|_{\CC^6\cap E_s}$ will have Jordan type $(2)$ or $(1,1)$. \acom{This example breaks. Why? How should we choose weights?}

Take 2: Let's try different weights.
\[
\begin{aligned}
    \mu_1 &= (1,1,0) & \lambda_1 &= (2,0,0) \\
    \mu_2 &= (1,1,1) & \lambda_2 &= (2,1,0) \\
    \mu &= (2,2,1) & \lambda &= (4,1,0)
\end{aligned}    
\]
and 
\[
\tau(1,0,0) = \young(12) \qquad \tau(1,0,1) = \young(12,3) \quad \tau(0,1,0) = \young(13,2)    
\]
Then 
\[
A = \left[\begin{BMAT}(e){cc;cc;c}{cc;cc;c} 
    0 & 1 & 0 & 0 & 0\\
    0 & s & A_0 & A_1 & A_2\\
    0 & 0 & 0 & 1 & 0\\
    0 & 0 & 0 & s & A_3\\
    0 & 0 & 0 & 0 & s
    \end{BMAT}\right]
\]
We have $E_0 \cap \CC^2 = \Sp(e_1)$, $E_s \cap \CC^2 = \Sp(e_1 + se_2)$. Next $E_0 \cap \CC^4$ is spanned by a 2-cycle generated by $-\frac{A_0} s e_2 + e_3$ and $E_s \cap \CC^4$ is spanned by a 2-cycle generated by 
$$
\frac{s}{A_1 + \frac{A_0}{s}}e_4 + \frac{1}{A_1 + \frac{A_0}{s}} e_3 - \frac 1 s e_1 
$$ 
or the additional 1-cycle $e_4 - \frac{A_1}{A_0} e_3$ assuming $A_0 + sA_1 = 0$. 
Finally $E_s$ is spanned by an additional 1-cycle 
$$
e_5 - \frac{A_2}{A_1 + \frac{A_0}{s}} e_4 + \frac 1 s \frac{A_2}{A_1 + \frac{A_0}{s}}e_3 
$$
assuming $A_3 = 0$.
% and $A_2 = 0$?? 
Or the two 1-cycles are extended to a 2-cycle and a 1-cycle, the 2-cycle generated by $-\frac{1}{s} e_1 + \frac s {A_2} e_5$.
% and again assuming $A_3 = 0$ in addition to $A_0 + sA_1 = 0$??
This gives us 
\[
    \young(12) \ast \young(12,3) = (A_3) \qquad % A_2, 
    \young(12) \ast \young(13,2) = (A_0 + sA_1) \to (A_0) %  , A_3
\]  
Does it agree with what is expected on the module/cluster side? 
\[
S_1 \ast (1\to 2) = S_1 \oplus (1 \to 2) \qquad S_1 \ast (1\leftarrow 2) = S_1 \oplus (1 \leftarrow 2)
\]
The MV cycle of $\young(12)$ is a $\PP^1$: via MVy it has an open subset comprised of matrices 
\[
    \left[\begin{BMAT}(e){c;c}{c;c} 
        0 & A_0\\
        0 & 0
        \end{BMAT}\right] : A_0\ne 0
\]
The MV cycles of the other two tableaux are made up of matrices of the form 
\[
    \left[\begin{BMAT}(e){c;c;c}{c;c;c} 
        0 & A_0 & A_1\\
        0 & 0 & A_2\\
        0 & 0 & 0
        \end{BMAT}\right] : 
        \begin{cases}
            A_0 \ne 0 \text{ and } A_2 = 0 & \tau = \young(12,3) \\
            A_0 = 0 \text{ and } A_2 \ne 0 & \tau = \young(13,2)
        \end{cases} 
\] both $\CC^2$'s. \acom{How do the coordinates relate?}
% This means we'll be extending the 2-cycle $\{e_2, e_1\}$ to a 3-cycle, and adding a 1-cycle to $E_0\cap\CC^3$. The new 1-cycle must be $\{e_4\}$. \acom{Why?} Hence our first equation: $A_0 = 0$. As to the 3-cycle, it can only be generated by $e_5$, i.e.\ the original 2-cycle is modified: $\{e_5 \mapsto A_1 e_3 + e_4 \mapsto \}$ 
% TODO: Make a table 
% 
% The zero eigenspace of $A$ conforms to the shape 
%     \[
%     \lambda_1 = \yng(3,1,1)    
%     \] 
%     is made of three cycles
%     \begin{align*}
%         a_2 e_1 &\leftarrow a_2e_2 - s e_4 \leftarrow a_2 e_3 - s e_5 \\
%         e_4 & \\
%         e_7 & 
%     \end{align*}
%     while the $s$-eigenspace confirming to the shape 
%     \[
%     \lambda_2 = \yng(2,1)    
%     \]
%     is made of the two cycles 
%     \begin{align*}
%         e_1 + se_2 +s^2 e_3 & \leftarrow e_2 + 2se_3 + e_7 + se_8 \\
%         e_4 + s e_5 + s^2 e_6 & 
%     \end{align*}
%     assuming $b_2 = s$ and $a_2 + sa_3 = 0$.
\end{example}
\newpage
\begin{example}[Roger]
Let
    \[
    \begin{aligned}
        \lambda_1 &= (2,0,0,0) & \mu_1 &= (1,1,0,0) \\
        \lambda_2 &= (2,2,1,0) & \mu_2 &= (3,2,1,1)
    \end{aligned}
    \]
so $\lambda_1 - \mu_1 = \alpha_1$ and $\lambda_2 - \mu_2 = \alpha_2 + \alpha_3$. We have the following young tableaux:
\[
\tau_1 = \young(12) \hspace{5mm}
\tau_2 = \young(11,23,4) \hspace{5mm}
\tau_2' = \young(11,24,3)
\]
where $\tau_1$ corresponds to the module $S_1$, $\tau_2$ corresponds to the module $2\rightarrow 3$, and $\tau'_2$ corresponds to the module $2 \leftarrow 3$.

The matrix we are considering is 
\[
X = \begin{bmatrix}
    0 & 1 \\
     & 0 & 1 \\
     & -s^2 & 2s & a & b & c & d \\
     & & & 0 & 1 \\
     & & & & s & e & f \\
     & & & & & s & g \\
     & & & & & & s
\end{bmatrix}
\]
such that $\dim E_0 = 2$, $\dim \ker X = 1$, $\dim E_s = 5$, and $\dim \ker (X-sI) = 3$ where $E_0$ and $E_s$ are the $0$- and $s$-generalized eigenspaces.

We see that the two-cycle in $E_0$ is 
\[
\left\{ X\left(e_2 + \frac{s^2}{a}e_4\right), e_2 + \frac{s^2}{a}e_4 \right\}
= \left\{e_1, e_2 + \frac{s^2}{a}e_4 \right\}.
\]

As $\tau_2$ and $\tau'_2$ both share $\young(11,2)$, we can find a 2-cycle from just the upper-left $3\times3$ block, and an additional vector in $\ker(X-sI)$ from the upper-left $5\times5$-submatrix. The 2-cycle from the $3 \times 3$ block is
\[
\left\{e_1 + se_2 + s^2e_3, -\frac{2}{s}e_1 - e_2 \right\}.
\]
The additional vector in $\ker(X-sI)$ is $e_4 + se_5$ and this requires $a+sb=0$.

Now consider the case that the young diagram we are working with is $\tau_2$. Then we have $e_4 + se_5 + x(e_1 + se_2 + s^2e_3)$ part of a 2-cycle that can be found by looking at the upper-left $6 \times 6$-submatrix. We find that the 2-cycle is
\[
\left\{e_4 + se_5 + x(e_1 + se_2 + s^2e_3), -\frac{2x}{s}e_1 -xe_2 -\frac{1}{s}e_4 + \frac{s}{e}e_6 \right\}
\]
and this requires that $ae - s^2c = 0$.

The last vector in $\ker(X-sI)$ comes from the entire $X-sI$ and we see it is $-fe_6 + ee_7$, which requires $g=0$ and $ed-cf=0$.

For the case $\tau'_2$, we start with find the third vector in $\ker(X-sI)$ from the upper-left $6 \times 6$-submatrix. We see that it is $e_6$, which requires $c = 0$ and $e = 0$. 

For the remaining 2-cycle, we want it to end with $x(e_1 + se_2 + s^2e_3) + (e_4 + se_5) + ye_6$ so our 2-cycle is
\[
\left\{ x(e_1 + se_2 + s^2e_3) + (e_4 + se_5) + ye_6, -\frac{2x}{s}e_1 -xe_2 -\frac{1}{s}e_4 +\frac{s}{f} e_7\right\}
\]
which requires $af-ds^2=0$ and $fy-sg = 0$. As $y$ is free, the last equation is not really a restriction on $f$ and $g$.

From the minimal polynomial, we have $X^2 (X-sI)^2 = 0$ which gives us the equations
\[
a+sb = cs+eb = bf+cg+ds = esg = 0.
\]
Taking $s\rightarrow 0$, we have the following equations for our two cases of $\tau_2$ and $\tau'_2$:
\[
\begin{array}{c|c}
    \tau_2 & \tau'_2 \\ \hline
    a = 0 & a = 0 \\
    g = 0 & c = 0 \\
    eb = 0 & e = 0 \\
    bf = 0 & bf = 0 \\
    ed-cf = 0 & 
\end{array}
\]

For the $\tau_2$ case, the coordinate ring is 
\[
\frac{\CC[a,b,c,d,e,f,g]}{\langle a,g,eb,bf,ed-cf \rangle}
\cong \frac{\CC[b,c,d,e,f]}{\langle eb,bf,ed-cf \rangle}
= \frac{\CC[b,c,d,e,f]}{\langle e,f \rangle \cap \langle b,ed-cf \rangle}
\]
Hence the associated algebraic set is reducible with two irreducible components. The component corresponding to the ideal $\langle e,f \rangle$ is $\AA^3$, which corresponds to $\PP^3$, while the ideal $\langle b,ed-cf \rangle$ corresponds to the toric variety whose toric polytope is a square-based pyramid. 

As $\tau_2$ corresponds to the module $2 \rightarrow 3$, the irreducible components should correspond to the modules $P_1 = 1 \rightarrow 2 \rightarrow 3$ and $1 \leftarrow 2 \rightarrow 3$. Indeed, the MV cycle corresponding to $P_1$ is the Grassmannian $Gr(1,4) \cong \PP^3$ and for $1 \leftarrow 2 \rightarrow 3$, we do get a toric variety with polytope the square-based pyramid.

However for the $\tau'_2$ case, the coordinate ring is 
\[
\frac {\CC[a,b,c,d,e,f,g]}{\langle a,c,e,df \rangle}
\cong \frac{\CC[b,d,f,g]}{\langle df \rangle}
\]
which corresponds to $\AA^3 \cup \AA^3$. Since $\tau'_2$ corresponds to the module $2\leftarrow 3$, we expect two irreducible components corresponding to the modules $P_3 = 1 \leftarrow 2 \leftarrow 3$ and $1 \rightarrow 2 \leftarrow 3$. $P_3$ corresponds to the variety $Gr(3,4) \cong \PP^3$ and $1 \rightarrow 2 \leftarrow 3$ also corresponds to a toric variety whose polytope is a square-based pyramid (?).
\end{example}

\begin{example}[Above example redone]
    In $\CC[A_0..A_6]$ where 
    \[
        \left[\begin{BMAT}(e){ccc;cc;c;c}{ccc;cc;c;c} 
            0 & 1 & 0 & 0 & 0 & 0 & 0\\
            0 & 0 & 1 & 0 & 0 & 0 & 0\\
            0 & -s^2 & 2s & A_0 & A_1 & A_2 & A_3\\
            0 & 0 & 0 & 0 & 1 & 0 & 0\\
            0 & 0 & 0 & 0 & s & A_4 & A_5\\
            0 & 0 & 0 & 0 & 0 & s & A_6\\
            0 & 0 & 0 & 0 & 0 & 0 & s
            \end{BMAT}\right]    
    \]
    we have 
    $$
    \begin{aligned}
        \young(12)*\young(11,23,4) &= ({\color{blue}A_6,A_2A_5-A_3A_4},A_1A_4 + A_2s,A_1s + A_0) \\
        &= (A_6,A_2A_5 - A_3A_4, A_1A_4,A_0) \text{ at } s = 0 \\
        &= (A_0,A_4,A_5,A_6) \sqcup (A_0,A_1,A_6,A_2A_5 - A_3 A_4) \sqcup (A_0,A_2,A_4,A_6)
    \end{aligned}
    $$ 
    while 
    $$
    \begin{aligned}
        \young(12)*\young(11,24,3) &= ({\color{blue}sA_0A_5 + (1 + s^2)A_1A_5+sA_3},A_2,A_4,A_1s + A_0) \\
        &= (A_1A_5,A_2,A_4,A_0) \text{ at } s = 0 \\
        &= (A_0,A_2,A_4,A_5) \sqcup (A_0,A_1,A_2,A_4) 
    \end{aligned} 
    $$
    Cycles are 
    \[
    \frac{A_0}{s^2}e_2 + e_4 \xrightarrow{A} e_1    
    \]
    and, in the $\young(11,23)$ case, 
    \[
    \begin{aligned}
        e_2 + 2s e_3 \xrightarrow{A-s} e_1 + se_2 + s^2 e_3 \\
        (sA_4 e_4 + (1 + s^2)A_4 e_5 + se_6)/(sA_4) \xrightarrow{A-s} (e_4 + se_5)/s 
    \end{aligned}    
    \]
    while in the $\young(11,2,3)$ case,
    \[
        \begin{aligned}
      e_6 \xrightarrow{A-s} 0 \\
      e_4 + (1/s + s) e_5 + (1/A_5) e_7 \xrightarrow{A-s} (1/s)e_4 + e_5 + \ast e_6   
        \end{aligned}
    \]
\end{example}
\end{document}