\documentclass{article}
\usepackage{basic}
\newcommand{\anne}[2]{\colorbox{pink!75!blue}{#1}\marginpar[]{\tiny\textbf{\color{pink!50!blue}#2}}}

\title{Working title: Mirkovi\'c--Vybornov fusion in Beilinson--Drinfeld Grassmannian}
% \title{mvybdcluster}
% \author{Roger Bai, Anne Dranowski, Joel Kamnitzer}
\date{October 2020}

\begin{document}

\maketitle

\section{Introduction}

The BD Grassmannian. The convolution Grassmannian. Distinguished orbits and slices. Connections to Mirkovi\'c--Vybornov \cite{mirkovic2007quiver,mirkovic2019comparison}, Cautis--Kamnitzer \cite{cautis2018categorical}, Anderson--Kogan \cite{anderson2005algebra}.

Address limitations outside of type $A$? 

\section{Notation}

In ordinary $\Gr$ we have the following lattice descriptions valid only in type \(A\).  
%   Let \(T\) be a maximal torus. 
%   Denote \(\cO = \CC[t]\) and \(\cK =\CC(t) \).  
  Given \(\mu\in X^\bullet(T)\), write \(t^\mu\) for its image in \(G(\cK)\) and 
  % 
  % E.g.\ \(t^\mu = \diag (t^{\mu_1},\dots,t^{\mu_m})\) in \(GL_m(\cK)\)  and 
  \(L_\mu\) for its image in 
  \[
    \Gr = G(\cK)/G(\cO)  \overset{A}{=}  \{L\overset{\text{free}}{\underset{\text{rank } m}{\subset}} \cO^m : tL \subset L \}
  \]   % (orbit-stabilizer).
% 
Example: \(L_\mu = \Sp_{\cO}(e_it^j : 0\le j <\mu_i)\).  
  Fact: \(\Gr^T = X^\bullet(T)\)  
  and other distinguished subsets (needed for the definition of MV cycles and later open subset thereof) are all orbits of fixed points  
  \[
    \begin{aligned}
  % \Gr &= G(\cK)/G(\cO) & & = \{L\subset \CC[t]^m : tL \subset L \} \\
  \Gr^\lambda &= G(\cO) L_\lambda & & = \{L\in\Gr : t\big|_{\cO^m/L} \text{ has Jordan type } \lambda\} \\
      \Gr_\mu & = G_1[t^{-1}] L_\mu & & = \{L\in \Gr : L = \Sp_{\cO}(v_1,\dots,v_m) \text{ such that } \\
        & & & \qquad\quad  v_j = t^{\mu_j} e_j + \sum p_{ij} e_i \text{ with  } \deg p_{ij} < \mu_j \} \\
  S^\mu_- &= U_-(\cK) L_\mu & & = \{L\in \Gr_\mu : \dim (\cO^k / L\cap \cO^k) = \mu_1 + \dots + \mu_k \}
    \end{aligned}
  \] 

Let $\Gr$ denote the ordinary \new{affine Grassmannian} $\Gch(\cK)/\Gch(\cO)$, $\Grbd$ the \new{Beilinson--Drinfeld affine Grassmannian}, and $\Grc$ the \new{convolution affine Grassmannian}. 

\begin{definition}
    The \new{BD Grassmannian} is the set 
    \begin{equation}
        \label{eq:grbd}
    \begin{aligned}
        \{(V,\sigma) : V &\text{ is a rank } m \text{ vector bundle on }\PP^1 \\ &\text{and } \sigma: V \dashrightarrow \scO_{\PP^1}^m \text{ is a trivialization} \\
        &\text{defined away from finitely many points in }\AA^1\}
    \end{aligned}
    \end{equation}
    The rank of the $m$ in the definition of $\Grbd$ is the dimension of the maximal torus of $\Gch$. For $\Gch =\GL_m = G$. 

    More generally, one can define a BD grassmannian of $\GL_m$ over any smooth curve $C$ as the reduced ind-scheme $\Grbd[n,C]$ fibered over a finite symmetric power of $C$ --- $C^{(n)}$ --- such that the fibre over the point $\vec p = (p_1,\dots,p_n)\in C^n$ is a collection of rank $m$ vector bundles $V$ over $C$ which are trivial away from $\vec p$ viewed also as a subset --- $\{p_1,\dots,p_n\}$ --- of $C$. Trivial means $\scO_C^m\cong V$. 
\end{definition}

\jcom{Careful: you can define it over $C^n$ or $C^{(n)}$, these are two different things.}

To quote \cite{baumann2020bases} the BD Grassmannian is a relative version of the affine Grassmannian where the base is the space of effective divisors on a smooth curve $C$. The choice $C = \AA^1$ ``amply satisfies our needs and offers three advantages: there is a natural global coordinate it, every \anne{$G$-torsor}{i.e. principal $G$ bundle?} on it is trivializable, and the monodromy of any local system is trivial. Formally, $\Grbd$ is the functor on the category of commutative $\CC$-algebras that assigns to an algebra $R$ the set of isomorphism classes of triples $(\vec p,V,\sigma)$ where $\vec p\in \AA^n(R)$, $V$ is a $G^\vee$-torsor over $\AA^1_R$ and $\sigma$ is a trivialization of $V$ away from $\vec p$.''
\jcom{Yes, $G$-torsor is the same thing as a principal $G$-bundle.}

They denote by $\pi$ the fibration $\Grbd\to\AA^n$ (forgetting $V$ and $\sigma$).
% 
Their simplified description is: it's the set of pairs $(\vec p,[\sigma])$ where $\vec p\in\CC^n$ and $[\sigma]$ is an element of the homogeneous space 
\[
G^\vee(\CC[z,(z-p_1)^{-1},\dots,(z-p_n)^{-1}])/G^\vee(\CC[z])    
\]
Their example, (almost) our setting: 
\begin{example}
    When $G = \GL_m\CC$ the datum of $[\sigma]$ is equivalent to the datum of the $\CC[z]$-lattice $\sigma(L_0)$ in $\CC(z)^m$ with $L_0 = \CC[z]^m$ denoting the \new{standard lattice}. Set $f_{\vec p} = (z - p_1)\cdots (z-p_n)$. Then a lattice $L$ is of the form $\sigma(L_0)$ if and only if there exists a positive integer $k$ such that $f^k_{\vec p}(L_0) \subseteq L\subseteq f^{-k}_{\vec p}(L_0)$ and for each $k$ they denote by $\Grbd_k$ the subset of $\Grbd$ consisting of pairs $(\vec p,L)$ such that this sandwhich condition holds. They identify $\CC[z]/(f_{\vec p}^{2k})$ with the vector space of polynomials of degree strictly less than $2kn$, and $L_0/f_{\vec p}^{2k}L_0$ with its $N$th product. Then 
    \[
    \Grbd_k \overset{\text{Zariski closed}}{\subset} \CC^n\times\bigcup_{d=0}^{2knN} G_d(L_0/f_{\vec p}^{2k}L_0)  
    \]
    where $G_d(?)$ denotes the ordinary Grassmann manifold of $d$-planes in the argument.
\end{example}

Our setting is $G= \GL_m$ and $n = 2$.

\begin{definition}
    The \new{deformed convolution Grassmannian} is [not needed?]
    pairs $(\vec p,[\vec\sigma])$ where $\vec p\in\CC^n$ and $\vec\sigma$ is in 
    \[
    G^\vee(\CC[z,(z-p_1)^{-1}]) \times^{G^\vee(\CC[z])}\cdots\times^{G^\vee(\CC[z])} G^\vee(\CC[z,(z-p_n)^{-1}])/G^\vee(\CC[z])
    \] with a map down to $\Grbd$ defined by $(\vec p,[\vec \sigma])\mapsto (\vec p,[\sigma_1\cdots\sigma_n])$. 
\end{definition}
To steal the follow-up example in \cite{baumann2020bases} where the above definition is also copied from\dots 
\begin{example}
    When $G = \GL_m\CC$ this deformation is described by the datum of $\vec p\in\CC^n$ and a sequence $(L_1,\dots,L_n)$ of $\CC[z]$-lattices in \anne{$\CC(z)^m$}{Why Laurent polynomials for the convolution?} such that for some $k\in\ZZ$ and for all $j \in \{1\dots n\}$
    \[
    (z-p_j)^kL_{j-1}\subset L_j\subset (z-p_j)^{-k} L_{j-1}    
    \]
    where again $L_0 = \CC[z]^m$ denotes the standard lattice, while $L_j = (\sigma_1\cdots\sigma_j)(L_0)$. Very nice. Very concrete. They can partition the deformation into \new{cells} by specifying the \new{relative positions} of the pairs $(L_{j-1},L_j)$ in terms of \new{invariant factors}. 
\end{example}
To be continued: \cite{baumann2020bases} go on to describe the fibres of the \anne{composition}{Not to $\CC^{(n)}$? Or to $\CC$?}
\[
    \Grc \to \Grbd \to \CC^n = \AA^n_\CC 
\] 
their description may be helpful.

For $\mu\in P$ and $p\in\CC$ they define 
\[
    \tilde S_{\mu | p} = (z-p)^\mu N^\vee (\CC[z,(z-p)^{-1}]) = N^\vee (\CC[z,(z-p)^{-1}])(z-p)^\mu
\]
They note that $\CC\xT[z-p]$ is the completion of $\CC(z)$ at ``the place defined by p'' and identify $\CC\xt[z-p]$ with $\CC\xt[z]$ and $\CC\xT[z-p]$ with $\CC\xT[z]$. 

They claim that 
\[
    N^\vee (\CC[z,(z-p)^{-1}])/N^\vee (\CC[z]) \to N^\vee(\CC\xT[z-p])/N^\vee(\CC\xt[z-p]) \cong N^\vee(\cK)/N^\vee(\cO)
\]
is bijective, and that mapping $\Gr$ and multiplying by $(z-p)^\mu$ one gets 
\[
\tilde S_{\mu | p}/N^\vee(\CC[z])\cong S_\mu
\]
They go on to describe the fusion product (section 5.3) a probably worthwhile read. 

Going forward, we'll use $t$ to denote the coordinate on $\AA^1_\CC = \CC$ instead of $z$. Then by $t^\mu \in \Gch(\cK)$ we'll denote the point defined by the coweight $\mu \in \Hom(\CC^\times,T^\vee) = T^\vee(\cK)$ and by $L_\mu$ its image $t^\mu \Gch(\cO)$ in $\Gr$.

\begin{definition}
    Given $\mu_1,\mu_2$ such that $\mu = \mu_1 +\mu_2$ is a partition of $N$ we define $T_{\mu_1,\mu_2}\subset \mat_N$ to be the set of $\mu\times\mu$ block matrices that are zero everywhere except possibly in the last $\min(\mu_i,\mu_j)$ columns of the last row of the $\mu_i\times\mu_j$th block \textit{plus} the block diagonal matrix whose $\mu_i\times\mu_i$ diagonal block is the companion matrix of $t^{\mu_{1,i}}(t-s)^{\mu_{2,i}}$ for each $i\in\{1,2,\dots,m\}$. We call this set \new{name}.
\end{definition}

\begin{remark}
    While we limit ourselves to the case of dominant partitions, the definition above makes sense for arbitrary partitions.
\end{remark}
\begin{remark}
    Speak to whether or not this slice appears in \cite{mirkovic2007quiver}. We don't think it does. But its ``lift'' might. 
\end{remark}

\begin{definition}
    Given $\lambda_1,\lambda_2$ such that $\lambda = \lambda_1 + \lambda_2$ is a partition of $N$ we define $\OO_{\lambda_1,\lambda_2}$ to be the set of $N\times N$ \anne{\new{semi-nilpotent}}{Cute?} matrices $X$ with spectrum in $\{0,s\}$ for some $s\in\CC^\times$ such that 
    $X\big|_{E_0} \in \OO_{\lambda_1}$ and $(X-s)\big|_{E_S}\in\OO_{\lambda_2}$ meaning TODO. 
    % restricted to its generalized $0$ eigenspace $X$ is nilpotent of Jordan type $\lambda_1$ and restricted to its generalized $s$ eigenspace $X - s$...
\end{definition}

Correspondingly we have 
\begin{itemize}
    \item $W_{\mu_1,\mu_2} = \Gch_1 [t^{-1},(t-s)^{-1}]t^{\mu_1} (t-s)^{\mu_2} \Gch(\cO)$ as a subset of $\Grbd[2]$ where recall the subscript 2 records the fact that we are fixing two points $0,s\in\CC$
    \item Does $\Grbd[2]^{\lambda_1,\lambda_2}$ admit such an orbit description? As a fibration it is pairs $(V,\sigma)$ such that is trivialized away from $(0,s)$ by $\sigma$ and $\sigma$ has type $(\lambda_1,\lambda_2)$ --- the data of the trivialization is equivalent to the data of pairs of lattices $(L_1,L_2)$ such that $L_i \in \Gr^{\lambda_i}$ away from $0$ and $L_1 = L_2\in\Gr^{\lambda}$ at 0?  
    \item In the deformed convolution Gr we have the subset $\Grc[2]^{\lambda_1,\lambda_2}$ of pairs $((0,s),[\sigma_1,\sigma_2])$ (really want arbitrary $(p_1,p_2)$ in place of $(0,s)$ which is the specialization that we make when we work in $\Grbd[2]$ sort of?) with $\sigma_i\in\Gch(\CC[t])(t-p_i)^{\lambda_i}\Gch(\CC[z])$ but this is not important? 
\end{itemize}

\begin{question}
    Can we describe $W_{\mu_1,\mu_2}\cap \overline{\Gr^{\lambda_1,\lambda_2}}$ as the set of lattices $L$ such that 
    \[
        t^{\lambda_{1,1}}(t-s)^{\lambda_{2,1}}L_0 \subset L \subset t^{-\lambda_{1,1}}(t-s)^{-\lambda_{2,1}}L_0
    \]
    and $\lim_{s\to 0} \rho^\vee (s)\cdot L = L_\mu$
\end{question}

\jcom{This is not correct since, you don't use all information in $ \lambda_1, \lambda_2$.  You need to describe the Jordan type of the quotients, or something equivalent.}

\begin{definition}
    Say $\mu_1$ and $\mu_2$ are \new{disjoint} if $(\mu_1)_i\ne 0 \Rightarrow (\mu_2)_i = 0$ and $(\mu_2)_i\ne 0 \Rightarrow (\mu_1)_i = 0$. 
\end{definition}

% \acom{I propose ``anodyne'' as another candidate for the above property after Kapranov--Shechtman.}

\section{Main results}

\begin{claim}
$\widetilde{T_x^a}\to\pi^{-1}(\overline{\Gr^\lambda}\cap\Gr_\mu)$ (this does depend on $b$! we get something like a springer fibre where the action of [what] on either side has eigenvalues a permutation of $b$.)
\end{claim}

\begin{claim}
Let $\cW^\mu_{\rm BD} = G_1\xT[t^{-1}]t^\mu$. Then $S^{\mu_1 + \mu_2}$ is contained in $\cW^\mu_{\rm BD}$ if $\mu$ is dominant. \jcom{And $\mu_1$, $\mu_2$ are dominant also?} \acom{Roger has a proof.}
\end{claim}

\begin{claim}
Let $a = (0,s)$ and suppose $\mu_1$ and $\mu_2$ are disjoint \sout{``transverse''} 
% i.e.\ $(\mu_1)_i\ne 0 \Rightarrow (\mu_2)_i = 0$ and $(\mu_2)_i\ne 0 \Rightarrow (\mu_1)_i = 0$. 
Let $\mu = \mu_1 + \mu_2$. Then $X\in\widetilde{T_x^a}$ is a $\mu\times\mu$ block matrix, with $(\mu_1)_k\times(\mu_1)_k$ diagonal block conjugate to a $(\mu_1)_k$ Jordan block and $(\mu_2)_k\times (\mu_2)_k$ diagonal block conjugate to $(\mu_2)_k$ Jordan block plus $sI$.
\end{claim}

\begin{question}
If $\mu_i$ is not a permutation of $\lambda_i$ and $\lambda_i$ are not ``homogeneous'' how do we proceed? E.g.\ if $\mu_1 = (3,0,2)$, $\mu_2 = (0,2,0)$ and $\lambda_1 = (4,1)$, $\lambda_2 = (2,0,0)$. 
\end{question}

\begin{question}
If $\mu_1$ and $\mu_2$ are not disjoint how do we proceed? E.g.\ if $\mu_1 = (2,2,0)$, $\mu_2 = (1,0,2)$; $\mu_1 = (2,2,1)$, $\mu_2 = (1,0,1)$.
\end{question}

\jcom{I think that we figured this out.  We just need to take the matrix corresponding to $ z^{\mu_1}(z-s)^{\mu_2} $ under the MVy isomorphism.}
\section{Convolution vs BD}
% aka, our result vs MVy's results
% aka, what they do, and what they stop short of doing

Fix $G = \GL(U) \cong \GL_m\CC$ and $\{e_1,\dots,e_m\}$ a basis of $U$. 
% 
Recall $\Gr = G(\cK) / G(\cO)$ where $\cK,\cO$\dots

\begin{definition}[Beilinson--Drinfeld loop Grassmannians]
    Denoted $\Grbd_{C^{(n)}}$ with $C$ a smooth curve (or formal neighbourhood of a finite subset thereof) and $C^{(n)}$ its $n$th symmetric power. It is a reduced ind-scheme $\Grbd_{C^{(n)}}\to C^{(n)}$ with fibres of $C$-lattices $\Grbd_b = \{(b,\cL) : b \in C^{(n)}\}$ made up of vector bundles \anne{such that}{Not sure what $\cO_C$ means} $\cL \cong U\otimes\cO_C$ off $b$ (i.e.\ over $C - \underline b$). The standard \anne{lattice}{Notation} is the pair $(\varnothing,\cL_0)$ with $\cL_0 = U\otimes \cO_C$. 
\end{definition}

{\bf The case $n = 1$.} Fix $b\in C$ and $t$ a choice of formal parameter. \anne{Then}{Why is this called ``its group-theoretic realization''} $\Grbd_b\cong\Gr$.

Furthermore, in this case, $C$-lattices $(b,\cL)$ are identified with $\cO$-submodules $L = \Gamma(\hat b, \cL)$ of $U_\cK = U\otimes\cK$ such that $L\otimes_\cO\cK \cong U_\cK$. 

Under this identification, we associate to a given $\lambda\in\ZZ^m$ the lattice (a priori a $\cO$-submodule) $L_\lambda = \oplus_1^m t^{\lambda_i}e_i\cO$. Nb. our lattices will be contained in the standard lattice $L_0$ whereas MVy's lattices contain.

Connected components of $\Gr$ are 

% Distinguished orbits. 
$G(\cO)$-orbits are indexed by coweights $\lambda = (\lambda_1\ge\lambda_2\ge\cdots\ge\lambda_n)$ of $G$. In terms of lattices 
\begin{equation}
    \label{eq:grlambdalat}
    \Gr^\lambda = \left\{L\supset L_0 \,\big|\, t\big|_{L/L_0} \in \OO_\lambda \right\}
\end{equation} in the connected component $\Gr_N$ are indexed

\cite{mirkovic2007quiver} define a map 
\begin{equation}
    \label{eq:grbdingrc}
    \Grbd \to \Grc
\end{equation}

\begin{itemize}
    \item Their slice $T_x$ or $T_\lambda$
    \item Their embedding $T_x\to\mathfrak G_N$
    \item $N$-dim $D$
    \item The map $\tilde {\bf m}:\tilde{\mathfrak g}^{n}\to\End(D)$
    \item The map ${\bf m} : \tilde{\mathcal N}^n\to\mathcal N$ sending $(x,F_\bullet)$ to $x$
    \item The map $\pi: \tilde{\mathfrak G}^n\to \mathfrak G$ sending $\mathcal L_\bullet$ to $\mathcal L_n$
\end{itemize}


The special case $b = \vec 0$. In this case $0$ in the affine quiver variety goes to the point $L_\lambda$ in the affine Grassmannian, and the preimage of zero in the smooth quiver variety ($=$ the core?) is identified with the preimage of $L_\lambda$ in the BD Grassmannian. 
% 
\[
\begin{tikzcd}
    \mathfrak L(\vec v,\vec w) \ar[r] \ar[d] & \pi^{-1} (L_\lambda)\ar[d] \\
     0 \ar[r] & L_\lambda
\end{tikzcd}    
\]

MVy write: ``we believe that one should be able to generalize this to arbitrary $b$'' and that's where we come in!

Recall the Mirkovi\'c--Vybornov immersion \cite[Theorems 1.2 and 5.3]{mirkovic2007quiver}. 

% \acom{All this time it was only an immersion! So our result will be a Theorem and not simply a reformulation plus corollary. Never mind.}

\begin{theorem}(\cite[Theorem 1.2 and 5.3]{mirkovic2007quiver})
    There exists an algebraic immersion $\tilde\psi$ 
    $$\widetilde{\bf m}^{-1}(T_\lambda)\cap\tilde\g^{n,a,E,\tilde\mu}\xrightarrow{\tilde\psi}\tilde{\mathfrak G}^{n,a}_{b}(P)$$
\end{theorem}

% Where to begin? How to tell this story? Left to right? 

% Here
% \begin{itemize}
%     \item $T_\lambda \equiv \TT_\lambda$
%     \item 
% \end{itemize}

\section{Statements and Proofs of Results}
\acom{Maybe split for now into a Notation section and a Proofs section}

Define
\[
S_{\mu_1, \mu_2} = N((t^{-1}))t^{\mu_1}(t-s)^{\mu_2}
\]
and
\[
W_\mu = G_1 [[t^{-1}]]t^\mu.
\]
Let $|\lambda| = |\lambda_1 + \lambda_2|$ and $|\mu| = |\mu_1 + \mu_2|$.

\acom{Why not $\lambda = \lambda_1 + \lambda_2$ and recall $\lvert \nu \rvert$ in general.}

\begin{lemma}[Proof in Proposition 2.6 of KWWY]
Suppose $\mu$ is dominant. Then 
\[
N((t^{-1})) t^\mu = N_1[[t^{-1}]] t^\mu.
\]
\end{lemma}

\begin{lemma}
For dominant $\mu_1,\mu_2$, we have
\[
S_{\mu_1, \mu_2} \subset W_{\mu_1 + \mu_2}.
\]
\end{lemma}

\begin{proof}
We have
\[
\begin{split}
    S_{\mu_1, \mu_2} & = N((t^{-1}))t^{\mu_1}(t-s)^{\mu_2} \\
     & \subset T_1[[t^{-1}]] N((t^{-1})) t^{\mu_1} (t-s)^{\mu_2} \\
     & = T_1[[t^{-1}]] N_1[[t^{-1}]] t^{\mu_1} (t-s)^{\mu_2} \\
     & = B_1[[t^{-1}]] t^{\mu_1} (t-s)^{\mu_2} \\
     & = B_1[[t^{-1}]] t^{\mu_1 + \mu_2} \\
     & \subset G_1[[t^{-1}]] t^{\mu_1 + \mu_2} \\
     & = W_{\mu_1 + \mu_2}
\end{split}
\]
where $B_1[[t^{-1}]] t^{\mu_1} (t-s)^{\mu_2} = B_1[[t^{-1}]] t^{\mu_1 + \mu_2}$ since 
\[
\frac{t}{t-s} = 
1 + \frac{s}{t} + \frac{s^2}{t^2} + \cdots 
\in B_1[[t^{-1}]].
\]
\end{proof}

Define $\Gr^{\lambda_1, \lambda_2} \subset \Gr_{BD}$ to be the family with generic fibre $\Gr^{\lambda_1} \times \Gr^{\lambda_2}$ and 0-fibre $\Gr^{\lambda_1 + \lambda_2}$.

Define $\OO_{\lambda_1, \lambda_2}$ to be matrices $X$ of size $|\lambda| \times |\lambda|$ such that 
\[
    % \begin{cases}
        X\big|_{E_0} \in \OO_{\lambda_1} 
        % \\
        \text{ and }
        (X-sI)\big|_{E_s} \in \OO_{\lambda_2}
    % \end{cases}
\]

Let 
\[
\mu = (\mu^{(1)}, \mu^{(2)}, ..., \mu^{(n)}).
\]
Define $\TT_{\mu_1, \mu_2}$ to be $|\mu| \times |\mu|$ matrices $X$ such that $X$ consists of block matrices where the size of the $i$-th diagonal block is $|\mu^{(i)}| \times |\mu^{(i)}|$, for $1\leq i \leq n$. Each diagonal block is the companion matrix for $t^{\mu_1}(t-s)^{\mu_2}$. Each off-diagonal block is zero everywhere except possibly in the last $\min(\mu_i,\mu_j)$ columns of the last row. 

\begin{theorem}
We have an isomorphism
\[
    \overline{\Gr^{\lambda_1, \lambda_2}} \cap S_{\mu_1, \mu_2} \cong
    \overline{\OO_{\lambda_1, \lambda_2}} \cap \TT_{\mu_1, \mu_2} \cap \n.
\]
\end{theorem}

\acom{Rather, corollary?}

\begin{proof}
We will prove this similarly to how the usual Mirkovi\'c--Vybornov isomorphism is proven.
\begin{enumerate}[label = Step \arabic*:]
    \item Define a map $\TT_{\mu_1, \mu_2} \cap \cN \rightarrow G_1[t^{-1}, (t-s)^{-1}] t^{\mu_1} (t-s)^{\mu_2}$.
    % 
    $$
    A \mapsto t^{\mu_1} (t-s)^{\mu_2} + a(t, t-s) \mapsto (L_1 \subset L_2) : (t-s)\big|_{L_2/L_1} = A\big|_{E_s}  , t\big|_{L_1/L_0} = A\big|_{E_0}
    $$
    % 
    Question: 1. is the middle matrix similar to a block matrix? 2. is the composition of these maps some intermediate level of MVy's $\psi$'s 

    BD Gr as lattices? $(L_1,L_2)\in\Gr\times\Gr$ corresponds to $L$ such that $L\otimes\CC\xt\cong L_1\otimes\CC\xt$ and $L\otimes\CC\xt[t-s]\cong L_2\otimes\CC\xt[t-s]$ where $\otimes = \otimes_{\CC[t]}$ or $\otimes_{\CC[t-s]}$ respectively even though Roger believes $\CC[t] = \CC[t-s]$.  

    Proof of Step 1: Let $\mu_1 = (\mu_{1,1}, \mu_{1,2}, ..., \mu_{1,n})$, $\mu_2 = (\mu_{2,1}, ..., \mu_{2,n})$, and $\mu = (\mu^{(1)}, \mu^{(2)}, ..., \mu^{(n)})$. Consider the matrix 
% TODO: remember to uncomment Roger's matrix 
\begin{comment}
\[
\sbox0{$\begin{matrix}1&2&3\\0&1&1\\0&0&1\end{matrix}$}
A = \left[
\begin{array}{c|c|c|c|c}
    \vphantom{\usebox{0}}\makebox[\wd0]{\large $C_1$}& \begin{matrix} \\ \\ a_{12}^1 & ... & a_{12}^{\mu^{(2)}} \end{matrix} & \makebox[\wd0]{\cdots} & \makebox[\wd0]{\cdots} &  \begin{matrix} \\ \\ a_{1n}^1 & ... & a_{1n}^{\mu^{(n)}} \end{matrix} \\ \hline
     & \vphantom{\usebox{0}}\makebox[\wd0]{\large $C_2$} & \begin{matrix} \\ \\ a_{23}^1 & ... & a_{23}^{\mu^{(3)}}  \end{matrix} & \makebox[\wd0]{\cdots} & \makebox[\wd0]{\vdots} \\ \hline
     & & \vphantom{\usebox{0}}\makebox[\wd0]{\ddots} & \vphantom{\usebox{0}}\makebox[\wd0]{\ddots} & \makebox[\wd0]{\vdots} \\ \hline
     & & & C_{n-1} & \begin{matrix} \\ \\ a_{n-1,n}^1 & ... & a_{n-1,n}^{\mu^{(n)}} \end{matrix} \\ \hline
     & & & & \vphantom{\usebox{0}}\makebox[\wd0]{$C_n$}
\end{array}
\right]
\]
\end{comment}
where $C_i$ is the $\mu^{(i)} \times \mu^{(i)}$ companion matrix of $t^{\mu_{1,i}}(t-s)^{\mu_{2,i}}$. We send this matrix to the matrix 
\[
\begin{bmatrix}
    t^{\mu_{1,1}}(t-s)^{\mu_{2,1}} \\
    -\sum_{k=1}^{\mu^{(2)}} a_{12}^k t^{k-1} & t^{\mu_{1,2}}(t-s)^{\mu_{2,2}} \\
    \vdots & \ddots & \ddots \\
    -\sum_{k=1}^{\mu^{(n)}} a_{1n}^k t^{k-1} & \cdots & -\sum_{k=1}^{\mu^{(n)}} a_{n-1,n}^k t^{k-1} & t^{\mu_{1,n}} (t-s)^{\mu_{2,n}}
\end{bmatrix}
\]
\[
= \begin{bmatrix}
    1 \\
    p_{1,2}(t) & 1 \\
    \vdots & \ddots & \ddots \\
    p_{1,n}(t) & \cdots & p_{n-1,n}(t) & 1
\end{bmatrix}
\begin{bmatrix}
    t^{\mu_{1,1}}(t-s)^{\mu_{2,1}} \\
     & \ddots \\
     & & t^{\mu_{1,n}} (t-s)^{\mu_{2,n}}
\end{bmatrix}
\]
where 
\[p_{i,j}(t) = \frac{-\sum_{k=1}^{\mu^{(j)}} a_{i,j}^k t^{k-1}} {t^{\mu_{1,i}} (t-s)^{\mu_{2,i}}}\]
As $\mu_1$ and $\mu_2$ are dominant, we have $p_{i,j}(t) \rightarrow 0$ as $t \rightarrow \infty$ so this matrix is indeed in $G_1[t^{-1},(t-s)^{-1}]t^{\mu_1}(t-s)^{\mu_2}$.

    \item If $A \in \TT_{\mu_1,\mu_2} \cap \n$ then $A$ is sent to $(N_-)_1[t^{-1}, (t-s)^{-1}] t^{\mu_1} (t-s)^{\mu_2}$. \acom{Requires MVyBD!}

    \item Conversely, given $L \in W_{\mu_1 + \mu_2}$, want to show surjectivity.
\end{enumerate}
\end{proof}

\bibliographystyle{alpha}
\bibliography{mvybd}
\end{document}
